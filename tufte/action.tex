\hypertarget{outline-of-this-page}{%
\subsection{Outline of this page}\label{outline-of-this-page}}

\begin{itemize}
\tightlist
\item
  \href{http://peeragogy.github.io/action.html\#micro-quickstart-guide--to-copy-to-peeragogyorg-}{Micro
  Quickstart Guide -- to copy to Peeragogy.org ✓}
\item
  \href{http://peeragogy.github.io/action.html\#new-content-very-high-level-outline-of-v4-the-outline-itself-is-still-wip}{New
  content: Very High-Level Outline of v4 (the outline itself is still
  WIP)}
\item
  \href{http://peeragogy.github.io/action.html\#technology-stack}{Technology
  Stack}
\item
  \href{http://peeragogy.github.io/action.html\#old-introduction}{Old
  Introduction}
\item
  \href{http://peeragogy.github.io/action.html\#paragogical-action-review}{\emph{Paragogical}
  Action Review}
\item
  \href{http://peeragogy.github.io/action.html\#editorial-roles}{Editorial
  Roles}
\item
  \href{http://peeragogy.github.io/action.html\#license-to-sign}{License
  to sign}
\item
  \href{http://peeragogy.github.io/action.html\#rheingoldian-real-time-meeting-roles}{Rheingoldian
  Real Time Meeting Roles}
\item
  \href{http://peeragogy.github.io/action.html\#new-content-peeragogical-innovations-9-week-pilot}{New
  content: Peeragogical Innovations (9 week pilot)}
\item
  \href{http://peeragogy.github.io/action.html\#old-content-welcome-to-the-peeragogy-accelerator}{Old
  content: Welcome to the Peeragogy Accelerator}
\end{itemize}

\hypertarget{micro-quickstart-guide-to-copy-to-peeragogy.org}{%
\subsection{Micro Quickstart Guide -- to copy to Peeragogy.org
✓}\label{micro-quickstart-guide-to-copy-to-peeragogy.org}}

\begin{itemize}
\tightlist
\item
  Our ongoing public discussions are on Google Groups at
  https://groups.google.com/forum/\#!forum/peeragogy. Sign up there to
  get email and post to the wider contributor community. Remember to ask
  questions!
\item
  The easiest way to get an orientation to our editing work is to join a
  live discussions on Mondays at 8PM UTC on Jitsi, at the following URL:
  https://meet.jit.si/peeragogy. Installation instructions for Jitsi are
  here: https://jitsi.org/downloads/
\item
  To comment on the Peeragogy Handbook please make an account at
  https://hypothes.is/ and use the mini-toolbar on the right-hand side
  of each page on peeragogy.org.
\item
  If you browse to https://github.com/Peeragogy you will see our Github
  ``organization''. The master copy of the Handbook content is at
  https://github.com/Peeragogy/Peeragogy.github.io. Github has a
  learning curve, ask for help.
\end{itemize}

Please have a look at this
\href{https://github.com/Peeragogy/peeragogy-handbook/wiki/Quickstart-guide}{longer
quickstart guide} for more information about our tools and workflow.

\hypertarget{new-content-very-high-level-outline-of-v4-the-outline-itself-is-still-wip}{%
\subsection{New content: Very High-Level Outline of v4 (the outline
itself is still
WIP)}\label{new-content-very-high-level-outline-of-v4-the-outline-itself-is-still-wip}}

Also there's a lot of similarity to the main steps in the Paragogical
Action Review:

Meso-handbook: Write a pattern for each of these major sections, 5 pages
long in total!

Each of the following bullet points should introduce something fairly
practical.

\begin{enumerate}
\def\labelenumi{\arabic{enumi}.}
\tightlist
\item
  \emph{Convene}. Review the intention: what do we expect to learn or
  make together? \textbf{Problem}

  \begin{itemize}
  \item
    \emph{Mini-introduction to Peeragogy.}
  \item
    \emph{What problem does peeragogy solve? Some history, where did the
    project come from?}
  \item
    \begin{itemize}
    \tightlist
    \item
      maybe putting a map in, to give some context -- like a concept map
      (ask Howard, find others)
    \end{itemize}
  \item
    Our intention is to write a ``How To Handbook'' (how does this help
    address the problem).
  \item
    How we have put things together here and how we are using the
    content\ldots{}
  \item
    Initial content could be based on
    \href{http://peeragogy.github.io/pattern-peeragogy.html}{Peeragogy}
    pattern
  \item
    Or on the
    \href{https://docs.google.com/document/d/1w2JZhpkrYYKknpJSSJgz23PPYxI31Cu1eWvw8I9ZraM/edit}{Starter
    Pack}.
  \item
    Could also incorporate a summary of the ``Convene'' section.
  \end{itemize}
\item
  \emph{Organize}. Establish what is happening: what and how are we
  learning?

  \begin{itemize}
  \tightlist
  \item
    Incorporate a summary of the ``Organize'' section
  \item
    \href{https://hackmd.io/LvcaTX1pTESFTtAMXK8lIg}{Newcomer}
  \item
    \href{https://hackmd.io/Z-ME-AU2R-203F31uig12A}{Heartbeat}
  \end{itemize}
\item
  \emph{Cooperate}. What are some different perspectives on what's
  happening?

  \begin{itemize}
  \tightlist
  \item
    \emph{Incorporate a summary of the ``Cooperate'' section}
  \item
    \href{https://hackmd.io/1n-ksWSyQvOw-x6vomBohg}{Carrying Capacity} -
    \emph{what is a different term for this} Ideal Size?
  \item
    \href{https://hackmd.io/hEZiRQPkS02BZzwtWJcsKQ}{Reduce, Reuse,
    Recyle}
  \end{itemize}
\item
  \emph{Assess}. What did we learn or change?

  \begin{itemize}
  \tightlist
  \item
    Incorporate a summary of the ``Assess'' section
  \item
    \href{https://hackmd.io/tnyTuPcaR_GtHZNnYcZyxA}{Landscape}
  \item
    \href{https://hackmd.io/Hz9Q3NU8Rgittp9b6oezHw}{Scrapbook} - where
    are we going to put our random thoughts! This can be more
    historical?
  \end{itemize}
\item
  \emph{Share}. What else should we change going forward? \textbf{What's
  Next}

  \begin{itemize}
  \tightlist
  \item
    Come up with a ``Share'' strategy and summarize it here.
  \item
    \href{https://hackmd.io/q5K5GstZTsqXTHrCRyYQJA}{Wrapper}
  \item
    \href{https://hackmd.io/xV24x23vQ2G1ScRHXBdMFA}{Specific Project}
  \end{itemize}
\item
  Index of Keywords from across the book

  \begin{itemize}
  \tightlist
  \item
    Keywords, glossary, similar terms and fields. Automatically generate
    this once we know what the keywords are.
  \end{itemize}
\end{enumerate}

\begin{itemize}
\tightlist
\item
  (was: \emph{Amanda's image, not terribly relevant but I do think we
  need some visuals - Charlotte ( how to put this on a separate line?)})
\item
  Joe: I've replaced that one because I wasn't sure of the licensing
  status. Remember to use public domain images! Various examples can be
  found on the
  \href{https://digitalcollections.nypl.org/search/index?filters\%5Brights\%5D=pd\&keywords=}{NYPL's
  website} (and, of course, elsewhere).
\end{itemize}

\textbf{Pattern template}

\begin{itemize}
\item
  \emph{Motivation} for using this pattern.
\item
  \emph{Context} of application.
\item
  \emph{Forces} that operate within the context of application, each
  with a mnemonic glyph.
\item
  \emph{Problem} the pattern addresses.
\item
  \emph{Solution} to the problem.
\item
  \emph{Rationale} for this solution.
\item
  \emph{Resolution} of the forces, named in bold.
\item
  \emph{Example 1} How the pattern manifests in current Wikimedia
  projects.
\item
  \emph{Example 2} How the pattern could inform the design of a future
  university.
\item
  \emph{What's Next} in the Peeragogy Project: How the pattern relates
  to our collective intention in the Peeragogy project
\item
  \emph{Many details to be added!}

  \begin{itemize}
  \tightlist
  \item
    See
    \href{https://docs.google.com/document/d/1v2TxWlYKqXuD2USl1Sb1OzCknZzTHjli1QCn7RrAQek/edit}{v4
    draft outline} on Google Docs for a more detailed draft outline.
    This version will become more concrete as we work.
  \item
    See
    \href{https://docs.google.com/spreadsheets/d/1pUlzk5uRYHdQmcM1pmllNKhvr21NH-ZXeZf2QJOyobw/edit\#gid=0}{Tufts
    Course} spreadsheet for one possible schedule of readings.
  \end{itemize}
\item
  \emph{Quickstart Guide}

  \begin{itemize}
  \tightlist
  \item
    https://github.com/Peeragogy/peeragogy-handbook/wiki/Quickstart-guide
  \item
    Live editable here: https://hackmd.io/syvktfQSTHmCcdYzwKWlEg
  \end{itemize}
\item
  \emph{Introductory Material}

  \begin{itemize}
  \tightlist
  \item
    Foreword, Preface, Introduction, Workbook
  \end{itemize}
\item
  \emph{Pattern Catalogue}
\end{itemize}

\hypertarget{technology-stack}{%
\subsection{Technology Stack}\label{technology-stack}}

\begin{itemize}
\tightlist
\item
  We store the master copy of the Handbook in Markdown on Github, in
  \href{https://github.com/Peeragogy/Peeragogy.github.io}{\textbf{this
  repository}}.
\item
  Github publishes to HTML on http://peeragogy.github.io/
  (peeragogy.org) redirects to that.
\item
  We are trialing live editing via Floobits, see
  \href{https://floobits.com/Peeragogy/Handbook/file/action.md:1?new_workspace=1}{\textbf{this
  page}}.
\item
  We have also been experimenting with similar features on HackMD, see
  \href{https://hackmd.io/zEY9rv5QR3O9JFl4jVCYFw}{\textbf{this page}}.
\item
  Downsteam processing has historically been via pandoc and LaTeX, in
  \href{https://github.com/Peeragogy/peeragogy-handbook}{\textbf{this
  repository}}.
\item
  There is an experimental tool for generating EPUB, in
  \href{https://gitlab.com/skreutzer/peeragogy-handbook-experimental}{\textbf{this
  repository}}
\end{itemize}

\hypertarget{old-introduction}{%
\subsection{Old Introduction}\label{old-introduction}}

\begin{quote}
We live where no one knows the answer and the struggle is to figure out
the question. {[}1{]}
\end{quote}

Welcome to the Peeragogy Handbook! We want to kick things off with a
candid confession: we're not going to pretend that this book is perfect.
In fact, it's not an ordinary book at all. The adventure starts when you
get out your pen or pencil, or mouse and keyboard, and begin marking it
up. It gets kicked into high gear when you join Peeragogy in Action.
You'll find a lot of friendly support as you write, draw, or dance your
own peeragogical adventure. But first, what is peeragogy?

Peeragogy is a flexible framework of techniques for peer learning and
peer knowledge production. Whereas pedagogy deals with the transmission
of knowledge from teachers to students, peeragogy is what people use to
produce and apply knowledge together. The strength of peeragogy is its
flexibility and scalability. The learning mind-set and strategies that
we are uncovering in the Peeragogy project can be applied in classrooms,
hackerspaces, organizations, wikis, and interconnected collaborations
across an entire society.

The Peeragogy Handbook is a compendium of know how for any group of
people who want to co-learn any subject together, when none of them is
an expert in the particular subject matter -- learning together without
one traditional teacher, especially using the tools and knowledge
available online. What we say in the Handbook draws extensively on our
experiences working together on the Handbook -- and our experiences in
other collaborative projects that drew us here in the first place. The
best way to learn about peeragogy is to do peeragogy, not just read
about it. Towards that end, coauthors and fans of the Handbook have an
active Google+ community, conveniently called Peeragogy in Action. We
maintain a regular schedule of weekly meetings that you're welcome to
join. The Handbook includes a short syllabus, which also called
``Peeragogy in Action'', and you can work through this with your own
group as you read through the book.

You're warmly invited to combine your local projects with the global
effort, and get involved in making the next edition of the Handbook.
That doesn't necessarily require you to do extensive writing or editing.
We're always interested in new use cases, tricky problems, and
interesting questions. In fact, our view is that any question is a good
question.

Here are some of the ways in which the current edition of the Handbook
is not perfect. You're welcome to add to the list! These are places
where you can jump in and get involved. This list gives a sense of the
challenges that we face putting peeragogy into action.

\hypertarget{scrapbook-of-peeragogical-problems}{%
\subsection{Scrapbook of Peeragogical
Problems}\label{scrapbook-of-peeragogical-problems}}

\hypertarget{maintaining-a-list-of-useful-resources}{%
\paragraph{Maintaining a list of useful
resources}\label{maintaining-a-list-of-useful-resources}}

We include references and recommended reading in the Handbook, and there
are a lot more links that have been shared in the Peeragogy in Action
community. It's a ongoing task to catalog and improve these resources --
including books, videos, images, projects, technology, etc. In short,
let's ``Reduce, Reuse, Recycle''! As a good start, Charlotte Pierce has
been maintaining a spreadsheet under the heading ``survey'' in our
Google Drive.

\hypertarget{developing-a-really-accessible-diy-tool-kit}{%
\paragraph{Developing a really accessible DIY
tool-kit}\label{developing-a-really-accessible-diy-tool-kit}}

A short ``workbook'' containing interviews and some activities follows
this introduction, but it could be much more interactive. Amanda Lyons
and Paola Ricaurte made several new exercises and drawings that we could
include. A more developed workbook could be split off from the handbook
into a separate publication. It would be great to have something simple
for onramping. For example, the workbook could be accompanied by video
tutorials for new contributors.

Paola Ricaurte points out that a really useful book will be easy to
sell. For teachers interested in peeragogy, this needs to be something
that can be use in workshops or on their own, to write in, to think
through issues. We're partway there, but to improve things, we really
need a better set of activities.

The next time Paola or someone else uses the handbook or workbook to run
a workshop, she can say, ``turn to this page, let's answer this
question, you have 10 minutes.'' There are lots of places where the
writing in the handbook could be made more interactive. One technique
Paola and Amanda used was turning ``statements'' from the handbook into
``questions.''

\hypertarget{crafting-a-visual-identity}{%
\paragraph{Crafting a visual identity}\label{crafting-a-visual-identity}}

Amanda also put together the latest cover art, with some collaboration
from Charlotte using inDesign. A more large-scale visual design would be
a good goal for the 4th Edition of the book. Fabrizio Terzi, who made
the handbook cover art for the 1st Edition, has been working on making
our website more friendly. So, again, work is in progress but we could
use your help.

\hypertarget{workflow-for-the-4th-edition}{%
\paragraph{Workflow for the 4th edition}\label{workflow-for-the-4th-edition}}

We've uploaded the content of the book to Github and are editing the
``live'' version of the site in Markdown. For this and previous print
editions, we've converted to LaTeX. There are a number of workflow
bottlenecks: First, people need to be comfortable updating the content
on the site. Second, it would be good to have more people involved with
the technical editing work that goes into compiling for print. Remember,
when we produce an actual physical handbook, we can sell it. In fact,
because all co-authors have transferred their copyright in this book to
the Public Domain, anyone can print and sell copies, convert the
material into new interactive forms, or do just about anything with it.

\hypertarget{translations}{%
\paragraph{Translations}\label{translations}}

Translating a book that's continually being revised is pretty much a
nightmare. With due respect to the valiant volunteer efforts that have
been attempted so far, it might be more convenient for everyone involved
to just pay professional translators or find a way to foster a
multi-lingual authoring community, or find a way to create a more robust
process of collective translation. Ideas are welcome, and we're making
some small steps here. More on this below.

\hypertarget{next-steps-whats-the-future-of-the-project}{%
\paragraph{Next steps? What's the future of the
project?}\label{next-steps-whats-the-future-of-the-project}}

In short: If we make the Handbook even more useful, then it will be no
problem to sell more copies of it. That is one way to make money to
cover future expenses. It's a paradigmatic example for other business
models we might use in the future. But even more important than a
business model is a sense of our shared vision, which is why we're
working on a ``Peeragogy Creed'' (after the Taekwondo creed, which
exists in various forms, one example is {[}2{]}). No doubt you'll find
the first version on peeragogy.org soon! Chapter 7 contains a further
list of practical next steps for the project.

\hypertarget{references}{%
\paragraph{References}\label{references}}

\begin{enumerate}
\def\labelenumi{\arabic{enumi}.}
\tightlist
\item
  Joshua Schimel, 2012. ``Writing Science'', Oxford University Press.
\item
  Taekwondo Student Creed, World Martial Arts Academy,
  http://www.worldtaekwondo.com/handbook.htm
\end{enumerate}

\hypertarget{paragogical-action-review}{%
\subsection{\texorpdfstring{\emph{Paragogical} Action
Review}{Paragogical Action Review}}\label{paragogical-action-review}}

\hypertarget{thursday-23-january-2020}{%
\subsection{Thursday 23 January
2020}\label{thursday-23-january-2020}}

\begin{enumerate}
\def\labelenumi{\arabic{enumi}.}
\tightlist
\item
  Review the intention: what do we expect to learn or make together?

  \begin{itemize}
  \tightlist
  \item
    Run peeragogical course
  \item
    Get ready for the Tufts version
  \item
    produce some materials to share
  \end{itemize}
\item
  Establish what is happening: what and how are we learning?

  \begin{itemize}
  \tightlist
  \item
    Lots of regulars joined
  \item
    Plus Chris who brought new energy and ideas
  \item
    We recording something on Zoom (slightly late start)
  \end{itemize}
\item
  What are some different perspectives on what's happening?

  \begin{itemize}
  \tightlist
  \item
    Joe: I talked a lot
  \item
    Charlotte: Maybe \textbf{start} each session with choosing roles
  \end{itemize}
\item
  What did we learn or change?

  \begin{itemize}
  \tightlist
  \item
    acquainted with material of the course
  \item
    cobwebs blown
  \item
    Mondays at 8PM UK there is another hands-on session on Jitsi
  \end{itemize}
\item
  What else should we change going forward?

  \begin{itemize}
  \tightlist
  \item
    Back next week, Deeper Dive into Co-learning, Will share a short
    video
  \end{itemize}
\end{enumerate}

\begin{itemize}
\tightlist
\item
  ``παραγωγή'' means \emph{production}
\item
  Cf. Howard Rheingold author of ``They Have A Word for It''
\end{itemize}

\hypertarget{monday-27-january-2020}{%
\subsection{Monday 27 January 2020}\label{monday-27-january-2020}}

\begin{itemize}
\tightlist
\item
  We wanted to make a new outline of the Peeragogy Handbook, and do some
  pratical hands-on editing
\item
  We did the outline but ran into some technology problems
\item
  We made progress on reorganizing things, and we related patterns and
  longstanding ``mini-handbook'' idea
\item
  HackMD has a lot of problems. Having regular meetings is good! We are
  able to have a good back and forth with a forward trajectory,
  improving and refining.
\item
  Joe: new keyboard! Robert? Roland? Charles Blass - are they up for it,
  or is there a better time? Have trial with FLOOBITS. Could plan basic
  setup by Thursday, with tested. Maybe an hour on Tuesday evening UK
  time with Joe \& Ray.
\end{itemize}

\hypertarget{editorial-roles}{%
\subsection{Editorial Roles}\label{editorial-roles}}

\hypertarget{management}{%
\subsection{MANAGEMENT}\label{management}}

That includes chasing people who have promised chapters.

\hypertarget{content}{%
\subsection{CONTENT}\label{content}}

Another major task that we had slated is to produce more activities and
mini-handbooks. A related task is an increasing ``patternization'' of
the content. Some of the old chapters can be shortened and turned into
new design patterns or short narrative sidebars.

\hypertarget{direction}{%
\subsection{DIRECTION}\label{direction}}

The comments generated in the Augment reading group which will conclude
on Tuesday give lots of hints about possible changes and improvements.
Particular attention should be given to the introductory chapters.

\hypertarget{technical}{%
\subsection{TECHNICAL}\label{technical}}

Then there is the technical editing, and getting everything to look
nice. We had discussed possibly involving a professional designer, but
it doesn't look like we have the funds to pay anyone.

\hypertarget{operations}{%
\subsection{OPERATIONS}\label{operations}}

Another relevant role is running and facilitating meetings. It is pretty
remarkable that we have been having meetings in this project almost
weekly since 2012! Assuming we keep up that pace on the way to
publication we are talking about approximately 24 production meetings in
the first half of next year.

\hypertarget{marketing}{%
\subsection{MARKETING}\label{marketing}}

Another task that we have kind of fallen down on in the past is
marketing the book. I think that in recognition of the tremendous amount
of effort that everyone has been putting into this, we should step up
our game in this regard for the fourth edition.

\hypertarget{license-to-sign}{%
\subsection{License to sign}\label{license-to-sign}}

Navigate to https://github.com/Peeragogy/Peeragogy.github.io

And find this:
https://github.com/Peeragogy/Peeragogy.github.io/blob/master/license.md

Then submit an email like this:

\begin{quote}
I hereby waive all copyright and related or neighboring rights together
with all associated claims and causes of action with respect to this
work to the extent possible under the law.
\end{quote}

\hypertarget{rheingoldian-real-time-meeting-roles}{%
\subsection{Rheingoldian Real Time Meeting
Roles}\label{rheingoldian-real-time-meeting-roles}}

\begin{itemize}
\item
  http://peeragogy.github.io/realtime.html
\item
  \textbf{Wrapper:} Share what we do with a wider audience
\item
  \textbf{Notetaker:} Write down what people say
\item
  \textbf{Research:}
\item
  \textbf{Whiteboard:}
\item
  \textbf{Searchers:} search the web for references mentioned during the
  session and other resources relevant to the discussion, and publish
  the URLs in the text chat
\item
  \textbf{Contextualizers:} add two or three sentences of contextual
  description for each URL
\item
  \textbf{Summarizers:} note main points made through text chat.
\item
  \textbf{Lexicographers:} identify and collaboratively define words and
  phrases on a wiki page.
\item
  \textbf{Mappers:} keep track of top level and secondary level
  categories and help the group mindmapping exercise at the end of the
  session.
\item
  \textbf{Curators:} compile the summaries, links to the lexicon and
  mindmaps, contextualized resources, on a single wiki page.
\item
  \textbf{Emergent Agendas:} using the whiteboard for anonymous
  nomination and preference polling for agenda items, with voice, video,
  and text-chat channels for discussing nominations, a group can quickly
  set its own agenda for the real-time session.
\end{itemize}

\hypertarget{new-content-peeragogical-innovations-9-week-pilot}{%
\subsection{New content: Peeragogical Innovations (9 week
pilot)}\label{new-content-peeragogical-innovations-9-week-pilot}}

\begin{quote}
We started with Tufts in mind, but we have a small cohort for an online
pilot.
\end{quote}

\begin{itemize}
\tightlist
\item
  Charlotte Pierce of Pierce Press
\item
  Chris Meadows of
  https://www.co-op.ac.uk/pages/category/co-operative-university
\end{itemize}

\hypertarget{auditors}{%
\subsection{AUDITORS}\label{auditors}}

\begin{itemize}
\tightlist
\item
  Charlie Danoff
\item
  Jeff Munro/ACMI.tv (tentative)
\item
  Paola Ricuarte
\item
  Ray Puzio
\end{itemize}

\textbf{January 2020}

In this course students will work together to design new ways to address
the global demand for learning opportunities. Our primary textbook will
be the Peeragogy Handbook (currently in a 3rd edition). This text may be
of particular interest to students in the Department of Education and
the Institute for Global Leadership, however, the accompanying readings
are fundamentally interdisciplinary, and anyone from any discipline is
welcome. Participants will contribute to critical review, expository
writing, media production, and creative design. One outcome will be a
collaboratively produced Massive Open Online Course (MOOC) based on the
course materials. We will design and develop additional innovative
interventions. Peer learning will be practiced throughout, by tailoring
the syllabus, developing new ways of processing and presenting the
course material, through supportive peer feedback, and in collaborative
final projects.

Contact time each week will be divided into a recitation, a practicum,
and an open studio.

\begin{enumerate}
\def\labelenumi{\arabic{enumi}.}
\tightlist
\item
  Recitation will be 1 hour with two seminar-style presentations led by
  students, synthesizing a collection of papers or summarizing a book.
\item
  Practicum will be a 1 hour long workshop-style interaction where we
  discuss that week's material and the next steps in the associated
  research.
\item
  Open studio will be 1 hour of time each week to work collaboratively
  on projects, and will include guest lectures and other smaller group
  activities.
\end{enumerate}

Our strategy will be to use the Handbook as our primary read/write
knowledge base, and to draw on other relevant texts to build a shared
language. In order to cover a wide range of material, it is not
necessary or expected for every student to read every text, but
presentation is required for a passing grade. One of our aims is to
learn how to do more as a team than we could accomplish as a loose group
of individuals. Discussions will be recorded and shared online to
broaden access and engage a wider public. The readings will be
frontloaded: the last four weeks of the course will be devoted to the
design and prototyping of new interventions that can be developed
further after the course.

\hypertarget{learning-outcomes}{%
\subsection{Learning outcomes}\label{learning-outcomes}}

By the end of the course, students will be able to synthesize
interventions relevant to global economic challenges. They will gain
design and media production skills relevant to creating a Massive Open
Online Course. It is expected that students will also train the
affective dimensions of their engagement with difficult issues, by
practicing rigorous self-assessment and developing constructive feedback
for their peers. Specifically, students successfully completing the
course will build a portfolio of evidence that they can receive major
challenges with compassion, respond with an awareness of diverse needs,
value others' perspectives and voices, organize effective networks and
strategies, and characterize constructive collaborative efforts and ways
to support them.

\hypertarget{agenda}{%
\subsection{Agenda}\label{agenda}}

\begin{description}
\tightlist
\item[Wk 1: Introduction (Joe)]
Develop a collective intention. The course will involve a lot of
thinking about co-design and we will start by co-designing the
experience we will have together. We will update this Agenda or
``Roadmap'' as we work together. We will introduce and distribute
Rheingoldian ``roles'' for our co-learning as well as editorial roles
needed for co-producing the next edition of the Peeragogy Handbook, and
an Action Review template.
\item[Wk 2.]
A deeper dive into co-learning. Reading and even regurgitating is fairly
passive. So, to learn more, students can devise more interesting ways to
share the material they are engaging with. Each student will focus on
developing expertise in one or two specific learning areas (i.e.~digital
comments, peer production, volunteer mobilization.
\item[Wk 3]
Gain experience with agile project management. We will adapt the
Handbook's Pattern Catalogue and maintain a record of `next steps' to
feed back into our local project(s).
\item[Wk 4]
Develop a networking strategy: Who else should we involve in our
learning? We will start reaching out to other people to co-design final
outcomes for the class. We will review the ``Data Fair'' from Data
Science for Design as one way to organize such outreach, and discuss
which methods will work best for us. For example, students may
experiment with uploading text to Wikipedia and engage in discussions
there.
\item[Wk 5]
Develop and discuss research designs. What questions will we be
addressing? What problems will we be solving? What problems will we not
be solving? What are appropriate research methods?
\item[Wk 6]
Gain experience with dialogue-facilitation strategies. How do the
approaches to peer learning that we have been studying so far in the
course relate to each other? E.g., imagine a conversation between
Benkler and Alexander, or Ostrom and Batchelor: would they agree about
anything? Disagree? We will explore different facilitation strategies to
strategically prepare for the design phase in the final weeks of the
course, asking ``What would be a good design for peer learning in your
planned intervention?''
\item[Wk 7]
Understand technologies used in peer production and small-scale
collaborations. What additional tools and literacies will we need to
``contribute back'' during the rest of the course? What technologies do
the projects that we are developing need in order to work optimally? Do
these tools exist? What would do we need to learn or develop to bring
them into existence, or to use existing tools well?
\item[Wk 8]
Discuss the relationship between learning and social movements. How can
we contextualise the Peeragogy project relative to other initiatives?
Does the project itself have ``peers'' that it can learn from?
\item[Wk 9]
Put peeragogy within its social and historical context. What does the
past, present, and future of learning look like? What role does
peeragogy play in economic development and sustainability?
\end{description}

(Weeks 1-9 are paired with readings in the Peeragogy Handbook and
Readings from the list below.)

\hypertarget{ideas-for-final-collaborative-projects}{%
\subsection{Ideas for final collaborative
projects}\label{ideas-for-final-collaborative-projects}}

Some questions that end-of-term projects might address include the
following:

\begin{itemize}
\tightlist
\item
  Quantitative trends can be easily analysed, but how do we measure,
  e.g., whether our design patterns are actually useful? Does using the
  patterns produce a qualitative change in the group; e.g., do they lead
  to a feeling of happiness for participants? Can we understand and
  revise our thinking about collaboration using Christopher Alexander's
  fifteen principles from the Nature of Order?
\item
  Are we successfully inventing new ways of relating that address the
  needs of people with limited access to educational opportunity? What
  does the global need look like? What inventions and interventions are
  out there now? What's missing?
\item
  Can we extract re-usable patterns from the literature on MOOCS,
  crowdfunding and other collaborative or collective projects? What are
  the best ways we can ``scale up'' the Peeragogy project and this
  course? And/or, how would we make peeragogy a more effective practical
  approach for projects at the local scale?
\item
  Can we develop our strategy for translating our learning within the
  Peeragogy project to (and from) diverse audiences? Who else should we
  be talking to? What other projects are doing similar things?
\item
  Borrowing a technique from religious studies, we can ask: how does
  Peeragogy differ from other related approaches? Is the world ready for
  a global peer learning movement? What can peer learning contribute to
  ongoing peer production efforts and vice versa?
\end{itemize}

\hypertarget{assessment}{%
\subsection{Assessment}\label{assessment}}

Students should consider the list of intended Learning Outcomes in this
syllabus when working on their self-assessments. Michael Wride's
\emph{Guide to Self-Assessment} (2017) will be discussed on the first
day of class.

\begin{itemize}
\tightlist
\item
  \textbf{Maybe make a 60 second video to describe your envisioned or
  realized learning outcome.}
\end{itemize}

\hypertarget{course-team}{%
\subsection{Course Team}\label{course-team}}

Coordinator Joe Corneli (Contact details: holtzermann17@gmail.com,
Subject: Tufts course)

\hypertarget{potential-guests-to-be-added}{%
\subsection{Potential Guests (to be
added):}\label{potential-guests-to-be-added}}

\begin{itemize}
\tightlist
\item
  Puna-Rimam Ripiye
\item
  Yenn Lee
\item
  Mexico informal MOOC people?
\item
  Cooperative University people?
\end{itemize}

\hypertarget{textbook}{%
\subsection{Textbook}\label{textbook}}

J. Corneli, C. J. Danoff, C. Pierce, P. Ricaurte, and L. Snow MacDonald,
eds.~The Peeragogy Handbook. 3rd ed.~Chicago, IL./Somerville, MA.:
PubDomEd/Pierce Press, 2016. The latest version of the Handbook is
available for free on Peeragogy.org. A 4th Edition is in development for
publication on Public Domain Day, Jan.~1, 2021.

\hypertarget{additional-readings}{%
\subsection{Additional Readings}\label{additional-readings}}

(Pick one or two of these to present, or argue for a substitution.)

\begin{itemize}
\tightlist
\item
  Sher. Wishcraft: How to Get What You Really Want
\item
  Ralya. Unframed: The Art of Improvisation for Game Masters
\item
  Illich. Tools for Conviviality
\item
  Rosovsky. The University: An Owner's Manual
\item
  Ostrom, Understanding institutional diversity
\item
  \textbf{Alexander et al.~The Oregon Experiment, ``The City is Not a
  Tree''.} \textbf{(Ray)}
\item
  \textbf{Batchelor. After Buddhism, last chapter?} \textbf{(Ray)}
\item
  Benkler. Collective Intelligence
\item
  \textbf{Weber. The Success of Open Source (Chris)}
\item
  Unger. Knowledge Economy
\item
  Jacobs. Dark Age Ahead
\item
  Aber. The Sustainable Learning Community
\item
  \textbf{Hill. Essays on Volunteer Mobilization in Peer
  Production(Chris)}
\item
  Ranciere. The Ignorant Schoolmaster
\item
  Mulholland. Re-imagining the Art School
\item
  Hassan. The Social Labs Revolution
\item
  \textbf{Banathy. Designing Social Systems in a Changing World}
  \textbf{(Joe)}
\item
  \textbf{Freire. Pedagogy of Freedom: Ethics, Democracy, and Civic
  Courage} \textbf{(Paola)}
\item
  \textbf{de Filipe} \textbf{Governance in online communities}
  \textbf{(Charlie)}
\item
  \textbf{``The convergence of digital commons with local manufacturing
  from a degrowth perspective: Two illustrative cases'' Vasilis
  Kostakisa, Kostas Latoufis, Minas Liarokapisc, \& Michel Bauwens
  (Charlotte) (plus a few recommended readings I'd like to explore on
  this topic)}
\end{itemize}

\hypertarget{timetable}{%
\subsection{Timetable}\label{timetable}}

A representative timetable is presented in this spreadsheet:
http://bit.ly/2OItJNa This will be jointly revised during the first week
of class and kept up to date with any changes.

\hypertarget{meeting-times-and-locations}{%
\subsection{Meeting times and
locations}\label{meeting-times-and-locations}}

\begin{itemize}
\tightlist
\item
  Monday -- 1 hour hands on editing session, 8PM Jitsi
\item
  Thursday -- 1 hour discussion session, 2PM Zoom
\end{itemize}

\hypertarget{additional-organizational-details}{%
\subsection{Additional organizational
details}\label{additional-organizational-details}}

\hypertarget{expect-to-do-a-lot-of-reading-and-some-writing.}{%
\paragraph{Expect to do a lot of reading (and some
writing).}\label{expect-to-do-a-lot-of-reading-and-some-writing.}}

(This will be revised for pilot version.) 6 hours of homework each week
is the federally mandated minimum corresponding to 3 contact hours for
higher education courses in the US. If you read at a rate of 2 minutes
per page, you can cover 180 pages in this time. This means that you
could cover up to 1800{[}u{]} pages in 10 weeks. Since you will have
other tasks too, 1000-1500 pages is a reasonable estimate of how many
pages you might expect to read over the course of the semester. Since
the books that you will be responsible for presenting in Recitation are
generally much shorter, you are expected to take the initiative to find
and digest supplementary materials. You are encouraged to use a tool
like Zotero to log your reading and share your personal bibliography and
notes, and also to share summaries and analysis more widely, e.g., on
Wikipedia or in updates to the Peeragogy Handbook. Presenters are
invited to enrich the presentations in their Recitation sections as they
see appropriate.{[}v{]}

\hypertarget{the-recitation-and-practicum-will-be-recorded-and-disseminated}{%
\paragraph{The Recitation and Practicum will be recorded and
disseminated}\label{the-recitation-and-practicum-will-be-recorded-and-disseminated}}

We will ask for an appropriate waiver. Students should also sign the CC
Zero waiver in advance of making any Peeragogy Handbook contributions,
and agree to CC-By-SA for any Wikipedia contributions.

\hypertarget{final-projects-should-be-demonstrably-collaborative.}{%
\paragraph{Final projects should be demonstrably
collaborative.}\label{final-projects-should-be-demonstrably-collaborative.}}

Each student is responsible for their own one-page summary and
evaluation of their contributions.

\hypertarget{old-content-welcome-to-the-peeragogy-accelerator.}{%
\subsection{Old content: Welcome to the Peeragogy
Accelerator.}\label{old-content-welcome-to-the-peeragogy-accelerator.}}

The purpose of the \emph{Peeragogy Accelerator} is to use the power of
peer-learning to help build great organizations, projects, and to work
through specific challenges.

We will do this by investing time and energy, rather than money,
building a distributed community of peer learners, and a strongly vetted
collection of best practices. Our project complements others' work on
sites like
\href{https://en.wikiversity.org/wiki/Wikiversity:Main_Page}{Wikiversity}
and \href{https://www.p2pu.org/en/}{P2PU}, but with an applied flavor.
It is somewhat similar to \href{https://www.ycombinator.com/}{Y
Combinator} and other start-up accelerators or incubators, but we're
doing it the
\href{https://en.wikipedia.org/wiki/Commons-based_peer_production}{commons
based peer production} way.

Here, we present \emph{Peeragogy in Action}, a project guide in four
parts. Each part relates to one or more sections of our handbook, and
suggests activities to try while you explore peer learning. These
activities are designed for flexible use by widely distributed groups,
collaborating via a light-weight infrastructure. Participants may be
educators, community organizers, designers, hackers, dancers, students,
seasoned peeragogues, or first-timers. The guide should be useful for
groups who want to build a strong collaboration, as well as to
facilitators or theorists who want to hone their practice or approach.
Together, we will use our various talents to build effective methods and
models for peer produced peer learning. We've labeled the phases as
Stage 1 through Stage 4, because that's the schedule we use, but if
you're working through this on your own, you can choose your own pace.
Let's get started!

\hypertarget{stage-1.-set-the-initial-challenge-and-build-a-framework-for-accountability-among-participants.-1-3-weeks}{%
\section{Stage 1. Set the initial challenge and build a framework for
accountability among participants. (1-3
weeks)}\label{stage-1.-set-the-initial-challenge-and-build-a-framework-for-accountability-among-participants.-1-3-weeks}}

\emph{Activity} -- Come up with a plan for your work and an agreement,
or informal contract, for your group. You can use the suggestions in
this document as a starting point, but your first task is to revise the
outline we've developed so that it suits your needs. It might be helpful
to ask: What are you interested in learning? What is your primary
intended outcome? What problem do you hope to solve? How collaborative
does your project need to be? How will the participants' expertise in
the topic vary? What sort of support will you and other participants
require? What problems won't you solve?

\emph{Technology} -- Familiarize yourself with the collaboration tools
you intend to use (e.g.~a public wiki, a private forum, a community
table, social media, or something else). Create something in text,
image, or video form introducing yourself and your project(s) to others
in the worldwide peeragogy community.

\emph{Suggested Resources} -- The Peeragogy Handbook, parts I
(`\href{http://peeragogy.org/}{Introduction}') and II
(`\href{http://peeragogy.org/motivation/}{Motivation}'). For a succinct
theoretical overview, please refer to our literature review, which we
have adapted into a
\href{http://en.wikipedia.org/wiki/Peer_learning}{Wikipedia page about
`Peer learning'}.

\emph{Observations from the Peeragogy project} -- We had a fairly weak
project structure at the outset, which yielded mixed results. One
participant said: ``I definitely think I do better when presented with a
framework or scaffold to use for participation or content development.''
Yet the same person wrote with enthusiasm about being ``freed of the
requirement or need for an entrepreneurial visionary.''

\emph{Further Reading} -- Boud, D. and Lee, A. (2005). \emph{`Peer
learning' as pedagogic discourse for research education}. Studies in
Higher Education, 30(5):501--516.

\emph{Further Questions}: \textbf{What subject or skill does YOUR group
want to learn?} OR \textbf{What product or service does YOUR group want
to produce?}

\begin{itemize}
\tightlist
\item
  identify members \& subgroups
\item
  survey members: interests, motivations, skills, experience, time
\item
  other
\end{itemize}

\textbf{What learning theory and practice does the group need to know to
put together a successful peer-learning program?} OR \textbf{What
specific theory and research does the group need to meet production or
service goals?}

\begin{itemize}
\tightlist
\item
  who has gone before?
  (\href{http://peeragogy.github.io/practice.html}{\textbf{Reduce,
  Reuse, Recycle}})
\item
  similar groups \& organizations
\item
  best \& worst practices
\item
  other similar products, for production
\item
  proven success strategies
\item
  other
\end{itemize}

\hypertarget{stage-2.-bring-in-other-people-to-support-your-shared-goals-and-make-the-work-more-fun-too.-1-2-weeks}{%
\section{Stage 2. Bring in other people to support your shared goals,
and make the work more fun too. (1-2
weeks)}\label{stage-2.-bring-in-other-people-to-support-your-shared-goals-and-make-the-work-more-fun-too.-1-2-weeks}}

\emph{Activity} -- Write an invitation to someone who can help as a
co-facilitator on your project. Clarify what you hope to learn from them
and what your project has to offer. Helpful questions to consider as you
think about who to invite: What resources are available or missing? What
do you already have that you can build on? How will you find the
necessary resources? Who else is interested in these kinds of
challenges? Go through the these questions again when you have a small
group, and come up with a list of more people you'd like to invite or
consult with as the project progresses.

\emph{Technology} -- Identify tools that could potentially be useful
during the project, even if it's new to you. Start learning how to use
them. Connect with people in other locales who share similar interests
or know the tools. Find related groups, communities, and forums and
engage with others to start a dialogue.

\emph{Suggested resources} -- The Peeragogy Handbook, parts IV
(`\href{http://peeragogy.org/convening-a-group/}{Convening a Group}')
and V
(`\href{http://peeragogy.org/organizing-a-learning-context/}{Organizing
a Learning Context}').

\emph{Observations from the Peeragogy project} -- We used a strategy of
``open enrollment.'' New people were welcome to join the project at any
time. We also encouraged people to either stay involved or withdraw;
several times over the first year, we required participants to
explicitly reaffirm interest in order to stay registered in the forum
and mailing list.

\emph{Further Reading} -- Schmidt, J. Philipp. (2009). Commons-Based
Peer Production and education. Free Culture Research Workshop Harvard
University, 23 October 2009.

\emph{Further Questions}: \textbf{Identify and select the best learning
resources about your topic} OR \textbf{Identify and select the best
production resources for that product or service}

\begin{itemize}
\tightlist
\item
  published resources
\item
  live resources (people)
\item
  other
\end{itemize}

\textbf{What is the appropriate technology and communications tools and
platforms your group needs to accomplish their learning goal?} OR
\textbf{How will these participants identify and select the appropriate
technology and communications tools and platforms to accomplish their
production goal or service mission?}

\begin{itemize}
\tightlist
\item
  internal platforms \& tools including meeting spaces, connecting
  diverse platforms
\item
  external (public-facing) platforms \& tools
\item
  other
\end{itemize}

\hypertarget{stage-3.-solidifying-your-work-plan-and-learning-strategy-together-with-concrete-measures-for-success-to-move-the-project-forward.-1-3-weeks}{%
\section{Stage 3. Solidifying your work plan and learning strategy
together with concrete measures for `success' to move the project
forward. (1-3
weeks)}\label{stage-3.-solidifying-your-work-plan-and-learning-strategy-together-with-concrete-measures-for-success-to-move-the-project-forward.-1-3-weeks}}

\emph{Activity} -- Distill your ideas by writing an essay, making visual
sketches, or creating a short video to communicate the unique plans for
organization and evaluation that your group will use. By this time, you
should have identified which aspects of the project need to be refined
or expanded. Dive in!

\emph{Technology} -- Take time to mentor others or be mentored by
someone, meeting up in person or online. Pair up with someone else and
share knowledge together about one or more tools. You can discuss some
of the difficulties that you've encountered, or teach a beginner some
tricks.

\emph{Suggested resources} -- The Peeragogy Handbook, parts VI
(`\href{http://peeragogy.org/co-facilitation/}{Cooperation}'), VII
(`\href{http://peeragogy.org/assessment/}{Assessment}'), and at least
some of part II
(`\href{http://peeragogy.org/patterns-usecases/}{Peeragogy in
Practice}').

\emph{Observations from the Peeragogy project} -- Perhaps one of the
most important roles in the Peeragogy project was the role of the
`Wrapper', who prepared and circulated weekly summaries of forum
activity. This helped people stay informed about what was happening in
the project even if they didn't have time to read the forums. We've also
found that small groups of people who arrange their own meetings are
often the most productive.

\emph{Further Reading} -- Argyris, Chris. ``Teaching smart people how to
learn.'' Harvard Business Review 69.3 (1991); and, Gersick, Connie J.G.
``Time and transition in work teams: Toward a new model of group
development.'' Academy of Management Journal 31.1 (1988): 9-41.

\emph{Further Questions}: \textbf{What are your benchmarks for success
in your learning enterprise?} OR \textbf{What are your benchmarks for
success in your production enterprise or service organization?}

\begin{itemize}
\tightlist
\item
  survey members
\item
  pilot testing
\item
  formal assessment
\item
  consensus
\item
  other
\item
  what's next?
\end{itemize}

\hypertarget{stage-4.-wrap-up-the-project-with-a-critical-assessment-of-progress-and-directions-for-future-work.-share-any-changes-to-this-syllabus-that-you-think-would-be-useful-for-future-peeragogues-1-2-weeks.}{%
\section{Stage 4. Wrap up the project with a critical assessment of
progress and directions for future work. Share any changes to this
syllabus that you think would be useful for future peeragogues! (1-2
weeks).}\label{stage-4.-wrap-up-the-project-with-a-critical-assessment-of-progress-and-directions-for-future-work.-share-any-changes-to-this-syllabus-that-you-think-would-be-useful-for-future-peeragogues-1-2-weeks.}}

\emph{Activity} -- Identify the main obstacles you encountered. What are
some goals you were not able to accomplish yet? Did you foresee these
challenges at the outset? How did this project resemble or differ from
others you've worked on? How would you do things differently in future
projects? What would you like to tackle next?

\emph{Writing} -- Communicate your reflection case. Prepare a short
written or multimedia essay, dealing with your experiences in this
course. Share the results by posting it where others in the broader
Peeragogy project can find it.

\emph{Suggested resources} -- The Peeragogy Handbook, parts VIII
(`\href{http://peeragogy.org/resources/technologies/}{Technologies,
Services, and Platforms}') and IX
(`\href{http://peeragogy.org/resources/}{Resources}').

\emph{Observations from the Peeragogy project} -- When we were deciding
how to license our work, we decided to use CC0, emphasizing
`re-usability' and hoping that other people would come and remix the
handbook. At the moment, we're still waiting to see the first remix
edition, but we're confident that it will come along in due course.
Maybe you'll be the one who makes it!

\emph{`Extra credit'} -- Contribute back to one of the other
organisations or projects that helped you on this peeragogical journey.
Think about what you have to offer. Is it a bug fix, a constructive
critique, pictures, translation help, PR, wiki-gnoming or making a cake?
Make it something special, and people will remember you and thank you
for it.

\emph{Further reading} -- Stallman, Richard.
``\href{http://www.gnu.org/philosophy/shouldbefree.html}{Why software
should be free}'' (1992).

\emph{Further Questions}: Write your own!

\hypertarget{micro-case-study-the-peeragogy-project-year-1}{%
\subsection{\texorpdfstring{{Micro-}Case Study: The Peeragogy Project,
Year
1}{Micro-Case Study: The Peeragogy Project, Year 1}}\label{micro-case-study-the-peeragogy-project-year-1}}

Since its conception in early 2012, the Peeragogy Project has collected
over 3700 comments in our discussion forum, and over 200 pages of
expository text in the handbook. It has given contributors a new way of
thinking about things together. However, the project has not had the
levels of engagement that should be possible, given the technology
available, the global interest in improving education, and the number of
thoughful participants who expressed interest. We hope that the handbook
and this accompanying syllabus will provide a seed for a new phase of
learning, with many new contributors and new ideas drawn from real-life
applications.

We began with these four questions:

\begin{enumerate}
\def\labelenumi{\arabic{enumi}.}
\item
  \emph{How does a motivated group of self-learners choose a subject or
  skill to learn?}
\item
  \emph{How can this group identify and select the best learning
  resources about that topic?}
\item
  \emph{How will these learners identify and select the appropriate
  technology and communications tools and platforms to accomplish their
  learning goal?}
\item
  \emph{What does the group need to know about learning theory and
  practice to put together a successful peer-learning program?}
\end{enumerate}

\hypertarget{micro-case-study-the-peeragogy-project-year-2}{%
\subsection{\texorpdfstring{{Micro-}Case Study: The Peeragogy Project,
Year
2}{Micro-Case Study: The Peeragogy Project, Year 2}}\label{micro-case-study-the-peeragogy-project-year-2}}

10 new handbook contributors joined in the project's second year. We've
begun a series of weekly Hangouts on Air that have brought in many
additional discussants, all key people who can help to fulfil
peeragogy's promise. The handbook has been considerably improved through
edits and discussion. The next step for us is putting this work into
action in the \emph{Peeragogy Accelerator}.

\hypertarget{micro-case-study-the-peeragogy-project-year-3}{%
\subsection{\texorpdfstring{{Micro-}Case Study: The Peeragogy Project,
Year
3}{Micro-Case Study: The Peeragogy Project, Year 3}}\label{micro-case-study-the-peeragogy-project-year-3}}

We published our plans as ``Building the Peeragogy Accelerator'',
presenting it at OER14 and inviting feedback. In the run up to this, we
had been very active creating additional abstracts and submitting them
to conferences. However, despite our efforts we failed to recruit any
newcomers for the trial run of the Accelerator. Even so, piloting the
Accelerator with some of our own projects worked reasonably
well,\footnote{For an overview, see
  \url{http://is.gd/up_peeragogy_accelerator}.} but we decided to focus
on the handbook in the second half of the year. As the project's line-up
shifted, participants reaffirmed the importance of having ``no camp
counsellors.'' In the last quarter of 2014, we created the workbook that
is now presented in Part I, as a quickstart guide to peeragogy. We also
revised the pattern catalog, and used the revised format to create a
``distributed roadmap'' for the Peeragogy project -- featured in Chapter
7 of the third edition of the handbook.
