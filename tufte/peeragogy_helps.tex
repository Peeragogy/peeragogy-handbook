\subsection{Main Actor}

Neo, who is a hacker by night, and an office worker by day (and who
reads Baudrillard in his spare time).

\subsection{Main Success Scenario}

\begin{enumerate}
\item
  Neo lives in New York City, and works as a programmer in an office
  near Wall Street. His day-job involves finding patterns in market data
  (see Kevin Slavin's
  \href{http://www.ted.com/talks/kevin\_slavin\_how\_algorithms\_shape\_our\_world.html}{TED
  talk}).
\item
  He has been walking past
  \href{http://en.wikipedia.org/wiki/Zuccotti\_Park}{Zucotti Park} on
  his way home and more or less he finds this protest stuff annoying (he
  has other stuff on his mind). But one of these evenings, one of the
  protestors catches his attention (she's dressed rather
  strikingly\ldots{}). They talk a bit, and he comes away thinking about
  what she said:
  ``\href{http://www.nycga.net/files/2011/11/DeclarationFlowchart\_v2\_large.jpg}{All
  our grievances are interconnected.}'' What if all the solutions are
  interconnected too?
\item
  Night time: Neo becomes increasingly obsessed with this idea. He's
  pulling down lots of web pages from OWS activists, from companies,
  from government websites -- again, looking for patterns. What would it
  take for OWS folks to solve the problems they worry so much about?
\item
  He eventually stumbles across the idea of pæragogy and it works like
  the ``red pill'': it's possible to solve the problems but only by
  working together. It would be hard to engineer a social media platform
  that will actually help with this (OWS folks mostly use Tumblr and
  aren't necessarily all that technologically minded). But he starts
  working on a
  \href{http://campus.ftacademy.org/wiki/index.php/Free\_Technology\_Guild}{tool}
  that's geared towards learning and sharing skills, while working on
  real projects. At first, it's just hackers who are using the tool, but
  over time they adapt it for popular use. Then things start to get
  interesting\ldots{}
\end{enumerate}
