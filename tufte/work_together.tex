\subsection{Metacognition and mindfulness in peer learning}

Metacognition and mindfulness have to do with your awareness how how you
think, talk, participate, and attend to circumstances.

\begin{quote}
\textbf{Alan Schoenfeld}: What (exactly) are you doing? Can you describe
it precisely?~ Why are you doing it?~ How does it fit into the
solution?~ How does it help you?~ What will you do with the outcome when
you obtain it? {[}1{]}
\end{quote}

It can be particularly useful to apply this sort of ``meta awareness''
as you think about the roles that you take on in a given project, the
kind of contributions you want to make, and what you hope to get out of
the experience. These are all likely to change as time passes, so it's
good to get in the habit of reflection.

\subsection{Potential roles in your peer-learning project}

\begin{enumerate}
\itemsep1pt\parskip0pt\parsep0pt
\item
  Leader, Manager, Team Member, Worker
\item
  Content Creator, Author, Content Processor, Reviewer, Editor
\item
  Presentation Creator, Designer, Graphics, Applications
\item
  Planner, Project Manager, Coordinator, Attendee, Participant
\item
  Mediator, Moderator, Facilitator, Proponent, Advocate, Representative,
  Contributor
\end{enumerate}

\subsection{Potential contributions}

\begin{enumerate}
\itemsep1pt\parskip0pt\parsep0pt
\item
  Create, Originate, Research, Aggregate
\item
  Develop, Design, Integrate, Refine, Convert
\item
  Write, Edit, Format
\end{enumerate}

\subsection{Potential motivations}

\begin{enumerate}
\itemsep1pt\parskip0pt\parsep0pt
\item
  Acquisition of training or support in a topic or field;
\item
  Building relationships with interesting people;
\item
  Finding professional opportunities through other participants;
\item
  Creating or bolstering a personal network;
\item
  More organized and rational thinking through dialog and debate;
\item
  Feedback about performance and understanding of the topic.
\end{enumerate}

The process of shared reflection can prime a group for cohesion and
success.~ It can be tremendously useful to think about the motivations
of other participants, and how these can be jointly served.~ How can we
re-use the ``side-effects'' of individual and cooperative efforts in a
useful way?

\includegraphics{http://peeragogy.org/wp-content/uploads/2013/09/sengai-gibon.jpg}
A famous work in ink by Sengai Gibon (1750--1837)

\subsection{Two theories of motivation}

One of the most prominent thinkers working in the field of
\mbox{(self-)}motivation is Daniel Pink {[}2{]}, who proposes a theory of
motivation based on autonomy, mastery, and purpose, or, more colorfully:

\begin{enumerate}
\itemsep1pt\parskip0pt\parsep0pt
\item
  The urge to direct my life
\item
  The desire to get better at something that matters
\item
  The yearning to do something that serves a purpose bigger than just
  ``myself''
\end{enumerate}

There's clearly a ``learning orientation'' behind the second point: it's
not just a matter of ``fun'' --- the sense of achievement matters.~ But
fun remains relevant. Thomas Malone {[}3{]} specifically asked ``What
makes things fun to learn?'' His proposed framework for building fun
learning activities is also based on the three ingredients: fantasy,
challenge, and curiosity.

We can easily see how ``participation'' relates to ``motivation'' as
described above. When I can get useful information from other people, I
can direct my own life better. When I have means of exploring my dreams
by chatting then over and exploring some of the elements in a safe way,
I'm in a much better position to make something in reality. A solid
reputation that comes from being able to help others serves as a good
indicator of personal progress, a sign that one is able to deal with
greater challenges. Relationships provide the most basic sense of being
part of something bigger than oneself: et cetera.~ We'll say more about
these matters in the chapters on
\href{http://peeragogy.org/cowork/}{Cooperation}.

\subsection{References}

\begin{enumerate}
\itemsep1pt\parskip0pt\parsep0pt
\item
  Schoenfeld, A. H. (1987). What's all the fuss about metacognition? In
  A. H. Schoenfeld (Ed.), \emph{Cognitive science and mathematics
  education} (pp. 189-215). Hillsdale, NJ: Lawrence Erlbaum Associates.
\item
  Pink, D. (2011). \emph{Drive: The Surprising Truth About What
  Motivates Us}, Canongate Books Ltd
\item
  Malone, T.W. (1981). Toward a Theory of Intrinsically Motivating
  Instruction, \emph{Cognitive Science}, 4, pp. 333-369
\end{enumerate}
