\subsection{Different ways to analyze the learning process}

After doing some personal reflection on the roles you want to take on
and the contributions you want to make (as we discussed above), you may
also want to work together with your learning group to analyze the
learning process in more detail. There are many different phases,
stages, and dimensions - some simple and intuitive, others more complex
-- that you can use to help structure and understand the learning
experience: we list some of these below. (Detailed references are
collected in the recommended readings at the end of the book.)

\begin{enumerate}
\itemsep1pt\parskip0pt\parsep0pt
\item
  Forming, Norming, Storming, Performing (from Bruce Tuckman)
\item
  The ``five-stage e-moderating model'' (from Gilly Salmon)
\item
  I, We, Its, It (from Ken Wilber -- for an application in modeling
  educational systems, see {[}1{]})
\item
  Assimilative, Information Processing, Communicative, Productive,
  Experiential, Adaptive (from Martin Oliver and Gráinne Conole)
\item
  Guidance \& Support, Communication \& Collaboration, Reflection \&
  Demonstration, Content \& Activities (from Gráinne Conole)
\item
  Considered in terms of ``Learning Power'' (Ruth Deakin-Crick \emph{et
  al}.)
\item
  Multiple intelligences (after Howard Gardner)
\item
  The associated ``mental state'' (after Mihaly Csíkszentmihályi; see
  picture)
\end{enumerate}

{[}caption id=``'' align=``aligncenter''
width=``300''{]}\href{http://commons.wikimedia.org/wiki/File\%3AChallenge_vs_skill.svg}{\includegraphics{http://upload.wikimedia.org/wikipedia/commons/thumb/f/f6/Challenge_vs_skill.svg/300px-Challenge_vs_skill.svg.png}}
\href{http://commons.wikimedia.org/wiki/File\%3AChallenge_vs_skill.svg}{Challenge
vs. Skill}. By w:User:Oliverbeatson (w:File:Challenge vs skill.jpg)
{[}Public domain{]}, via Wikimedia Commons{[}/caption{]}

\subsection{Peer learning for one}

Can you apply the ideas of peer learning on your own? In a certain
sense, it's impossible, but somehow that never stops people from trying.
We find a striking parallel between the paragogy principles and the 5
Elements of Effective Thinking proposed by Edward Burger and Michael
Starbird in a recent book {[}2{]}. It's a nice short book and worth a
read. Here, we will just quote the titles of the main chapters:

\begin{enumerate}
\itemsep1pt\parskip0pt\parsep0pt
\item
  Quintessence, Engaging Change: Transform Yourself
\item
  Earth, Grounding Your Thinking: Understanding Deeply
\item
  Air, Creating Questions out of Thin Air: Be your own Socrates
\item
  Water, Seeing the Flow of Ideas: Look Back, Look Forward
\item
  Fire, Igniting Insights through Mistakes: Fail to Succeed
\end{enumerate}

We think that ``thinking'' is often most effective when it's done with
others, and this is something that Burger and Starbird don't give much
attention. Nevertheless, even when you find yourself on your own in the
midst of that challenging DIY project, you can use the techniques of
peer learning to understand yourself as a growing, changing part of a
shared context in motion. This can contribute to an effective and
adaptive outlook on life.

We invite you to approach this book as a ``peer learner'' -- and we hope
the techniques we've introduced here will serve you well in the world at
large.~~ The book, in part, documents the growth of our subject as it
moved from a critical and basically normative view to a richer
descriptive theory, rooted in a collection strategies for doing emergent
design.~ It's been fun -- and worthwhile (if also frustrating at times)
-- working on it.~ We sincerely hope you enjoy the rest of the book, but
don't be sparing with your criticism and creative ideas! You'll find
some further reflections on these matters in the sections on
\href{http://peeragogy.org/assessment/}{peeragogical assessment}.

\subsection{References}

\begin{enumerate}
\itemsep1pt\parskip0pt\parsep0pt
\item
  Corneli, J., and Mikroyannidis, A. (2012). Crowdsourcing education on
  the Web: a role-based analysis of online learning communities, in
  Alexandra Okada, Teresa Conolly, and Peter Scott (eds.), \emph{Collaborative
  Learning 2.0: Open Educational Resources}, IGI Global.
\item
  Burger, E. and Starbird, M. (2013). \emph{The 5 Elements of Effective
  Thinking}, Princeton University Press.
\end{enumerate}
