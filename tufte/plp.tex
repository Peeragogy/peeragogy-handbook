Author: Geoff Walker Editor: Joe Corneli

\subsection{\emph{How does a Personal Learning Plan (PLP) relate to
Peeragogy?}}

Redecker et al., (2009) indicate that social networks give rise to
innovation in teaching and learning by:

\begin{itemize}
\item
  Increasing the accessibility and availability of learning content;
\item
  Providing new formats for knowledge dissemination, acquisition and
  management;
\item
  Allowing for the production of dynamic learning resources and
  environments of high quality and interoperability;
\item
  Embedding learning in more engaging and activating multimedia
  environments;
\item
  Supporting individualized learning processes by allowing learner
  preferences to be accounted for;
\item
  Equipping learners and teachers with versatile tools for knowledge
  exchange and collaboration, which overcome the limitations of
  face-to-face instruction.
\end{itemize}
So, examination of these six areas could reveal how a PLP for
self-learning may be developed.

When we refer to content in social networks, we are usually referring to
User Generated Content (UGC). UGC is content generated by members of the
network for other network members. UGC is often seen as `conversational
media', but, it is often the case that `connectedness' is seen as more
important than `conversation'. If we are to use conversational media
appropriately and effectively we need to connect to content in ways in
which the connection and the content combine to produce conversation: a
kind of network triangulation. The micro-blog Twitter has become very
effective in making such a triangulation. It is essential that learners
have available and accessible content which reflects this triangulation
and PLP need to demonstrate how this triangulation is taking place in
any chosen social network.

Each learner's learning is shaped by a variety of influences --- for
example, family, cultural, peer, school, religious, local, and global
influences. The PLP supports each learner to shape plans and to achieve
success in education, the community, work and training. Through
participation in the program of learning, each learner can build into
his or her plan opportunities to:

\begin{itemize}
\item
  Identify and develop his or her capabilities;
\item
  Consider and access the range of learning options available, both
  inside the curriculum and externally, to develop and achieve personal
  learning goals;
\item
  Interact with a range of people with relevant expertise, including
  teachers, peers, mentors, and employers;
\item
  Learn from experience how to develop, implement, review, adjust, and
  achieve his or her goals and to plan and make decisions accordingly.
\end{itemize}
\subsection{\emph{Designing and developing a PLP for self-learning}}

A PLP is designed to develop a learner's learning and teaching
capabilities. Learners learn how to develop, implement, review, and
adjust personal learning goals. The PLP supports learners in developing
knowledge and skills that will enable them to:

\begin{enumerate}
\item
  Identify appropriate future options;
\item
  Review their strengths and areas for development;
\item
  Identify goals and plans for improvement;
\item
  Monitor their actions and review and adjust plans as needed to achieve
  their goals.
\end{enumerate}
\subsection{CASE STUDY: SUKDEV SINGH}

Suk wants to improve his social networking skills. Where should he start
as a self-learner? A PLP should be designed and developed which takes
account of the four areas above.

As a starting point, he needs to reflect on previous learning, in
particular, to understand his learning style and how this can be applied
to the self-learning process. An important starting point is to write a
learning autobiography which illustrates learning-to-date and projects
patterns of future learning. The autobiography should also include
details of relevant qualifications and details of key skills.

The next stage is to create a learning progression plan which could take
the form of a timeline. This would show where Suk would like to be, in
say, one month, three months, six months and a year.

Suk should now map out a support network to underline the self-learning
process. The network can include learning peers, professional
colleagues, friends and relatives.

Drawing upon learning style, past learning and projected patterns of
learning, he needs to design and develop a timetable to programme his
learning. A self-learning timetable should be divided into date, time,
location and session with space for personal evaluation of each session.

The PLP should also include a section for review and reflection and
comments of this kind should be written in such a way that they can be
shared for evaluation by both peers and mentors.

\subsection{Three key steps to be followed when drafting an appropriate
and effective PLP}

\emph{Key Step 1 - Learning needs:}

What do you most need to learn about in the time ahead?

\emph{Key Step 2 - Learning activities:}

What are the best ways you learn, what learning activities will meet
your learning needs, what help will you need and how long will it take?

\emph{Step 3 - Evidence of learning}:

What will you put into your personal portfolio to demonstrate your
learning progress and achievements?

\subsection{\emph{Questions for Step 1: Figure out your learning needs}}

KEY QUESTION:

What do you most need to learn about in the time ahead?

REMEMBER:

\begin{enumerate}
\item
  Build on any previous PLPs!
\item
  Focus on areas of weakness and not only about the things you are good
  at!
\item
  Think about all aspects of your work!
\item
  Include things which will raise your confidence and self-esteem!
\end{enumerate}
ASK YOURSELF:

\begin{enumerate}
\item
  Is there any need outstanding from your last plan or recent events?
\item
  What needs do I have arising from instances when my work has seemed
  difficult or less satisfactory?
\item
  What do I need to learn about to feel confident and fulfilled?
\end{enumerate}
\textbf{Make a note of your most important learning needs, then proceed
to step two.}

\subsection{\emph{Questions for step 2: figure out suitable learning
activities}}

KEY QUESTION: What are the best ways you can learn, what learning
activities will meet your learning needs, what help will you need and
how long will it take? REMEMBER:

\begin{enumerate}
\item
  Build on past experiences and consider a wide range of activities!
\item
  Pick the most appropriate activity for each need!
\item
  Include activities you are already doing regularly!
\item
  Be realistic about the time each activity will take and the help you
  will need!
\end{enumerate}
ASK YOURSELF:

\begin{enumerate}
\item
  How have I learnt best in the past, can I use methods which have
  worked well before?
\item
  What learning methods and activities are readily available to me?
\item
  Is the activity I have chosen appropriate?
\item
  How can activities I am already involved in, and wish to continue
  with, be incorporated into my PLP?
\item
  What help will I need and who will provide it?
\end{enumerate}
\textbf{Make a note of your chosen learning activities and number of
hours you think each will take.}

\subsection{\emph{Questions for Step 3: How will you demonstrate
evidence of learning?}}

KEY QUESTION:

What will you put into your portfolio to demonstrate your learning
progress and achievements?

REMEMBER:

\begin{enumerate}
\item
  Think about your learning and how you will do things differently in
  future!
\item
  Share some of the things you have learnt with your colleagues!
\item
  Look for ways that your learning has actually benefited others!
\item
  Organise the evidence you collect in a folder so that it can be
  presented when needed!
\end{enumerate}
ASK YOURSELF:

\begin{enumerate}
\item
  How will I show that I have benefited from my learning?
\item
  How will I show that others have benefited?
\end{enumerate}
\textbf{Make a note of your ideas about what evidence to collect.}
