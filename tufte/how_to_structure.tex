\subsection{From syllabus and curriculum to personal and peer learning
plans}

Part of the reason for the effectiveness of peeragogy is that the
``syllabus'' or ``curriculum'' -- more generally, the learning plan --
is developed by the people doing the learning. You won't faint with
shock when you see the reading list if you helped write it.

Having youwn own learning plan at the outset helps each participant
identify his or her unique learning and teaching proclivities and
capabilities, and effectively apply them in the peer setting. In
developing your personal plan, you can ask yourself the following
questions:

\begin{enumerate}
\itemsep1pt\parskip0pt\parsep0pt
\item
  What do I most need to learn about in the time ahead?
\item
  What are the best ways I learn, what learning activities will meet my
  learning needs, what help will I need and how long will it take?
\item
  What will I put into my personal portfolio to demonstrate my learning
  progress and achievements?
\end{enumerate}

Early in the process, the peer learning group should also convene to
develop a peer learning plan. In the Peeragogy project, we used live
meetings and forum-style platforms to discuss the group-level versions
of the questions listed above. Personal learning needs and skills were
also aired via these platforms, but the key shared outcome was an
initial project plan. Initially this took the form of an outline of
handbook chapters to write, as well as a division of labor.

Nothing was set in stone, and both the peer group and individual
participants have continued to develop, implement, review, and adjust
their goals as the project develops. We have stayed sufficiently
connected to the original goal of producing a handbook about peer
learning that you now have one in your hands (or on your screen). We've
also added some new goals for the project as time has gone by. Having a
malleable framework enables peer learners to:

\begin{enumerate}
\itemsep1pt\parskip0pt\parsep0pt
\item
  Identify appropriate directions and goals for future learning;
\item
  Review their strengths and areas for development;
\item
  Identify goals and plans for improvement;
\item
  Monitor their actions and review and adjust plans as needed to achieve
  their goals;
\item
  Update the goals to correspond to progress.
\end{enumerate}

This doesn't mean you have to let chaos rule, but often in the swirl of
ideas and contributions, new directions take shape and new ideas take
hold.

\subsection{Self-generating templates}

Documentation like mind maps, outlines, blogs or journals, and forum
posts for a peer learning project can create an audit trail or living
history of the process. You can mine the documentation of a past
peer-learning project or a completed phase of an ongoing project for
effective learning patterns, and if you're careful to document
everything, you can really benefit by taking the time to compare what
you've achieved against the stated goal or mission at the outset.~ Use
the record to reflect and evaluate key elements of the process for you
as a facilitator and as a member of the peer learning group.~ Adapt your
next phase of planning accordingly.

\subsection{From corporate training to learning on the job}

{[}caption id=``attachment\_1999'' align=``aligncenter''
width=``380''{]}\href{http://peeragogy.org/peer-learning-overview/learn-2/}{\includegraphics{http://peeragogy.org/wp-content/uploads/2012/12/learn.png}}
``I think because of the tremendous changes we see in education and at
work, the sets (attitudes) are beginning to overlap more and more,''
said Joachim Stroh of the Google+ community, Visual
Metaphors.{[}/caption{]}

Today's knowledge workers typically have instant, ubiquitous access to
the internet. The measure of their ability is an open-book exam. ``What
do you know?'' is replaced with ``What can you do?'' And if they get
bored, they can relatively easily be mentally elsewhere.

This has ramifications for the way managers manage as well as the way
teachers teach. To extract optimal performance from workers, managers
must inspire them rather than command them. Antoine de Saint-Exupéry put
it nicely: ``If you want to build a boat, do not instruct the men to saw
wood, stitch the sails, prepare the tools and organize the work, but
make them long for setting sail and travel to distant lands.''

\begin{quote}
\textbf{Jay Cross}: ``If I were an instructional designer in a moribund
training department, I'd polish up my resume and head over to marketing.
Co-learning can differentiate services, increase product usage,
strengthen customer relationships, and reduce the cost of hand-holding.
It's cheaper and more useful than advertising. But instead of just
making a copy of today's boring educational practices, build something
based on interaction and camaraderie, perhaps with some healthy
competition thrown in. Again, the emphasis should always be on learning
in order to do something!''
\end{quote}

In the section on \href{http://peeragogy.org/organize/}{organizing a
learning context}, we'll say quite a bit more about the implications
that our full conception of peer learning has for managers, teachers,
and other facilitators.
