Once more we're back to the question, ``What makes learning fun?'' There
are deep links between play and learning. Consider, for instance, the
way we learn the rules of a game through playing it. The first times we
play a card game, or a physical sport, or a computer simulation we test
out rule boundaries as well as our understanding. Actors and
role-players learn their roles through the dynamic process of
performance. The resulting learning isn't absorbed all at once, but
accretes over time through an emergent process, one unfolding further
through iterations. In other words, the more we play a game, the more we
learn it.

In addition to the rules of play, we learn about the subject which play
represents, be it a strategy game (chess, for example) or simulation of
economic conflict. Good games echo good teaching practice, too, in that
they structure a single player's experience to fit their regime of
competence (cf.~Vygotsky's zone of proximal learning, a la Gee
{{[}1{]}}). That is to say a game challenges players at a level suited
to their skill and knowledge: comfortable enough that play is possible,
but so challenging as to avoid boredom, eliciting player growth.
Role-playing in theater lets performers explore and test out concepts;
see Boal {{[}2{]}}. Further, adopting a playful attitude helps
individuals meet new challenges with curiousity, along with a readiness
to mobilize ideas and practical knowledge. Indeed, the energy activated
by play can take a person beyond an event's formal limitations, as
players can assume that play can go on and on {{[}3{]}}.

\begin{quote}
\textbf{Douglas Thomas and John Seely Brown}: ``All systems of play are,
at base, learning systems.'' {{[}4{]}}
\end{quote}

Games have always had a major social component, and learning plays a key
role in that interpersonal function. Using games to build group cohesion
is an old practice, actually a triusm in team sports.

It is important to locate our peeragogical moment in a world where
gaming is undergoing a renaissance. Not only has digital gaming become a
large industry, but gaming has begun to infiltrate non-gaming aspects of
the world, sometimes referred to as ``gamification.'' Putting all three
of these levels together, we see that we can possibly improve
co-learning by adopting a playful mindset. Such a playful attitude can
then mobilize any or all of the above advantages. For example,

\begin{itemize}
\tightlist
\item
  Two friends are learning the Russian language together. They invent a
  vocabulary game: one identifies an object in the world, and the other
  must name it in Russian. They take turns, each challenging the other,
  building up their common knowledge.
\item
  A middle-aged man decides to take up hiking. The prospect is somewhat
  daunting, since he's a very proud person and is easily stymied by
  learning something from scratch. So he adopts a ``trail name'', a
  playful pseudonym. This new identity lets him set-aside his
  self-importance and risk making mistakes. Gradually he grows
  comfortable with what his new persona learns.
\item
  We can also consider the \textbf{design} field as a useful kind of
  playful peeragogy. The person \emph{playing the role} of the designer
  can select the contextual frame within which the design is performed.
  This frame can be seen as the \emph{rules} governing the design, the
  artifact and the process. These rules, as with some games, may change
  over time. Therefore the possibility to adapt, to tailor one's
  activities to changing context is important when designing playful
  learning activities. (And we'll look at some ways to design peer
  learning experiences next!)
\end{itemize}

Of course, ``game-based learning'' can be part of standard pedagogy too.
When peers create the game themselves, this presumably involves both
game-based learning and peer learning. Classic strategy games like
\href{http://senseis.xmp.net/?MythOfOrigin}{Go} and
\href{http://www.amazon.com/Chess-Success-Using-Strengths-Children/dp/0767915682}{Chess}
also provide clear examples of peer learning practices: the question is
partly, what skills and mindsets do our game-related practices really
teach?

\begin{quote}
\textbf{Socrates}: ``No compulsory learning can remain in the soul
\ldots In teaching children, train them by a kind of game, and you will
be able to see more clearly the natural bent of each.''
\end{quote}

\hypertarget{exercises-that-can-help-you-cultivate-a-playful-attitude}{%
\subsection{Exercises that can help you cultivate a playful
attitude}\label{exercises-that-can-help-you-cultivate-a-playful-attitude}}

\begin{itemize}
\tightlist
\item
  Use the \href{http://www.rtqe.net/ObliqueStrategies/}{Oblique
  Strategies} card deck (Brian Eno and Peter Schmidt, 1st edition 1975,
  now available in its fifth edition) to spur playful creativity. Each
  card advises players to change their creative process, often in
  surprising directions.
\item
  Take turns making and sharing videos. This online collaborative
  continuous video storytelling involves a group of people creating
  short videos, uploading them to YouTube, then making playlists of
  results. Similar to \href{http://clipkino.info/}{Clip Kino}, only
  online.
\item
  Engage in theater play using Google+ Hangout. e.g.~coming together
  with a group of people online and performing theatrical performances
  on a shared topic that are recorded.
\end{itemize}

\hypertarget{references}{%
\subsection{References}\label{references}}

\begin{enumerate}
\def\labelenumi{\arabic{enumi}.}
\item
  Gee, J. P. (1992). \emph{The social mind: Language, ideology, and
  social practice}. Series in language and ideology. New York: Bergin \&
  Garvey.
\item
  Boal, A. (1979). \emph{Theatre of the oppressed}. 3rd ed.~London:
  Pluto Press.
\item
  Bereiter, C. and Scadamalia, M. (1993). \emph{Surpassing ourselves, an
  inquiry into the nature and implications of expertise}. Peru,
  Illinois: Open Court.
\item
  Douglas Thomas and John Seely Brown (2011), \emph{A New Culture of
  Learning: Cultivating the Imagination for a World of Constant Change}.
  CreateSpace.
\item
  Malone, T.W. (1981), Toward a Theory of Intrinsically Motivating
  Instruction, \emph{Cognitive Science}, 4, pp.~333-369
\end{enumerate}
