\section{Landmarks from the life of
peeragogy}\label{landmarks-from-the-life-of-peeragogy}

\subsection{Feedback from two novice course
organizers}\label{feedback-from-two-novice-course-organizers}

\emph{Before the Peeragogy project as such was convened, two of us
realised that ``peer produced peer learning'' could benefit from further
theoretical and practical development. Here is a summary of our early
thoughts as volunteer course organizers at the Peer-2-Peer University
(P2PU):}

▶ Our best experiences as course organizers happened when we were
committed to working through the material ourselves. Combining this with
gently prompting peers to follow through on their commitments could go a
long way towards keeping engagement at a reasonable level -- but this
only works when commitments are somewhat clear in the first place.

▶ It is typical for online communities to have strictly enforced
community norms. It would be helpful to have a concise discussion of
these available, together with up to date information on ``best
practices'' for organizers and participants. The current Course Design
Handbook provides one starting point, but it falls short of being a
complete guide to P2PU.

▶ In a traditional university, there are typically a lot of ways to
resolve problems without dropping out. P2PU's new ``Help Desk'' could
help with this issue -- if people use it.

▶ P2PU would have to work hard to use anything but ``participation'' as
a proxy value for ``learning.'' In terms of broader issues of quality
control, one serious thought is for P2PU core members (including staff)
to use the platform to organize their activities -- entirely in the
open.

▶ It is our firm belief that P2PU should work on a public roadmap that
leads from now up to the point where the vision is achieved. Both vision
and roadmap should be revised as appropriate.

\subsection{The ``FLOK Doc''}\label{the-flok-doc}

\emph{In 2013, Ecuador launched the Free/Libre/Open Knowledge Society
Project to facilitate the transition to a `\emph{buen saber}', or `good
knowledge' society, which is an extension of the official strategy
towards a `\emph{buen vivir}'-based society. The Peeragogy project
contributed a brief to help develop this plan. Here are some
highlights:}

Ecuador has a law about free software and open knowledge (\emph{Decreto
1014}, launched 2008). Article 32 of the \emph{Ley orgánica de Educación
Superior} makes open source software mandatory for higher education.
Public universities are building their own OER repositories. What
peeragogy can offer are are working methods for co-producing relevant
Open Educational Resources on a wider scale. As such, peeragogy is
especially relevant to the goals and working methods of the \emph{Human
Capabilities} stream of the FLOK project, but here are some ways it
could affect the other streams:

▶ \emph{Commons-oriented Productive Capacities} will require people to
learn new ways of working. Can we start to build a peeragogical
``extension school'', by collaborating on a new handbook about
sustainable agricultural techniques?

▶ \emph{Social Infrastructure and Institutional Innovation} will require
collaboration between many different agencies, local enterprises, and
global organizations. Can peeragogy help these groups cooperate
effectively? Coauthoring a handbook about inter-agency cooperation could
help.

▶ \emph{Hardware and Connectivity} needs to be connected to
documentation and active, participatory, support that shows how to use
and adapt new technologies to our use cases.

▶ \emph{Commons' Infrastructure for Collective Life} could co-develop
along with a pattern language that shows how to interconnect elements of
knowledge and practical solutions that are (re)generative of the commons
and relevant to learners' needs.

\subsection{New strategies for ``good faith
collaboration''}\label{new-strategies-for-good-faith-collaboration}

\emph{We're strongly in favor of the Wikimedia Foundation's mission,
``to empower and engage people around the world to collect and develop
educational content under a free license or in the public domain, and to
disseminate it effectively and globally.'' We hope peeragogy can
contribute to this and other free/open efforts to constructively reshape
the way education works in the future. Some values we share with the
Wikipedia project:}

▶ \emph{Neutral POV}: Pretty much anyone can write an article for the
\emph{Peeragogy Handbook} on anything related to peer learning and peer
production. We'll help review and edit to make the work shine. Rather
than requiring each individual article to be neutral, we strive for
overall comprehensiveness.

▶ \emph{Free content}: We've taken the radical step of putting material
in the handbook into the public domain, which means that anyone can
reuse material in the handbook for any purpose whatsoever, without
asking permission or even giving us attribution. The reason being: we
want to make re-use, application, and extension of this work as simple
as possible.

▶ \emph{Respect and civility}: We strive to focus on learning. If
someone disagrees with a given choice, we remember that in true dialogue
there are no right or wrong answers and no one in charge. If someone
seems to be frustrated with the way the project is going, we ask why and
attempt to learn from them about what we could change -- in order to
learn more.

▶ \emph{No firm rules}: The project roadmap is fluid, and our
understanding of the idea of ``peeragogy'' is revised and extended as we
go. The living patterns we catalog (in Part {[}practice-part{]}) aren't
prescriptive but they do seem to reappear with variation across
different learning scenarios. We don't have a fixed platform or
leadership structure, but use whatever tools and teams seem most
suitable for the purpose at hand.
