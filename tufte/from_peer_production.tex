\subsection{Main actor}

Julian, an enthusiastic convert to the power of peer-learning.

\subsection{Main success scenario}

\begin{enumerate}
\item
  Reflecting on the success of
  \href{http://socialmediaclassroom.com/host/peeragogy/forum/patterns-and-use-cases\#comment-1749}{Strategy
  as Learning}, Julian notes that other housing associations might
  benefit from this process. He also notes that as most housing
  association boards are made up of volunteers like himself, there is a
  very wide variation in background, knowledge and skills, and therefore
  not only a need for low cost (free) learning opportunities, but a
  range of skills available to enable them.
\item
  Julian sets up a peer learning resource on the web, drawing on the
  experiences in implementing
  \href{http://socialmediaclassroom.com/host/peeragogy/forum/patterns-and-use-cases\#comment-1749}{Strategy
  as Learning}, and promotes it through industry-specific web forums. He
  draws attention from an online journalist writing in the housing field
  who writes a positive article, and as a result a growing number of
  collaborators come forward.
\item
  Over a period of a year or so, the core team of active users
  collaborate to create standards and exemplars in relation to different
  aspects of housing association governance that become a de facto
  standard in the sector.
\end{enumerate}
\subsection{Thoughts}

\begin{enumerate}
\item
  Obviously a very specific use case that could easily be generalised
\item
  Possible patterns to extract? Seeding Peer Communities, Emergent
  Standards, Emergent Assessment ???
\end{enumerate}
