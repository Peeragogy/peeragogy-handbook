I use the idea of a \emph{pattern language} as a shorthand for what
Christopher Alexander talks about in his
\href{http://www.patternlanguage.com/archive/ieee/ieeetext.htm}{keynote
address} for the IEEE in 1996.

In short, once we have come up with enough patterns (including the
pattern of a \emph{pattern language} that I discussing here, and its
generalizations per Christopher Alexander), then we will be better able
to do both the socio-technical design work associated with planning
pæragogical experiences, and, quite likely, enjoy the ``actual work''
more too.

In this quote from the linked article, C. A. talks about computer
programming, but I think the same could go for any other sort of
design-and-implementation work:

\begin{quote}
\emph{It is a view of programming as the natural genetic infrastructure
of a living world which you/we are capable of creating, managing, making
available, and which could then have the result that a living structure
in our towns, houses, work places, cities, becomes an attainable thing.
That would be remarkable. It would turn the world around, and make
living structure the norm once again, throughout society, and make the
world worth living in again.This is an extraordinary vision of the
future, in which computers play a fundamental role in making the world -
and above all the built structure of the world - alive, humane,
ecologically profound, and with a deep living structure.}
\end{quote}
