This section about organizing Co-Learning rests on the assumption that
learning always happens in a context, whether this context is a
structured ``course'' or a (potentially) less structured ``learning
space''. For the moment we consider the following division:

\begin{itemize}
\tightlist
\item
  \emph{Organizing Co-learning Contexts}

  \begin{itemize}
  \tightlist
  \item
    Courses (``linked to a timeline or syllabus'')
  \item
    Spaces (``not linked to a timeline or syllabus'')
  \end{itemize}
\end{itemize}

This section focuses on existing learning contexts and examines in
detail how they have been ``organized'' by their .~ At a ``meta-level''
of development, we can talk about this parallel structure:

\begin{itemize}
\tightlist
\item
  \emph{Building Co-learning Platforms}

  \begin{itemize}
  \tightlist
  \item
    Development trajectories (e.g.~``design, implement, test, repeat'')
  \item
    Platform features (e.g.~forums, wikis, ownership models, etc.)
  \end{itemize}
\end{itemize}

A given learning environment will have both time-like and space-like
features as well as both designed-for and un-planned features. A given
learning platform will encourage certain types of engagement and impose
certain constraints. The question for both ``teachers'' and ``system
designers'' -- as well as for learners -- should be: \emph{what features
best support learning?}

The answer will depend on the learning task and available resources.

For example, many people believe that the best way to learn a foreign
language is through immersion. But not everyone who wants to learn, say,
French, can afford to drop everything to go live in a French-speaking
country. Thus, the space-like full immersion ``treatment'' is frequently
sacrificed for course-like treatments (either via books, CDs, videos, or
ongoing participation in semi-immersive discussion groups).

System designers are also faced with scarce resources: programmer time,
software licensing concerns, availability of peer support, and so forth.
While the ideal platform would (magically) come with solutions
pre-built, a more realistic approach recognizes that problem solving
always takes time and energy. The problem solving approach and
associated ``learning orientation'' will also depend on the task and
resources at hand. The following sections will develop this issue
further through some specific case studies.

\hypertarget{case-study-1-paragogy-and-the-after-action-review.}{%
\subsection{Case Study 1: ``Paragogy'' and the After Action
Review.}\label{case-study-1-paragogy-and-the-after-action-review.}}

In our analysis of our experiences as course organizers at P2PU, we (Joe
Corneli and Charlie Danoff) used the US Army's technique of After Action
Review (AAR). To quote from
\href{http://paragogy.net/ParagogyPaper2}{our paper} {{[}2{]}}:

\begin{quote}
As the name indicates, the AAR is used to review training exercises. It
is important to note that while one person typically plays the role of
evaluator in such a review {{[}\ldots{}{]}} the review itself happens
among peers, and examines the operations of the unit as a whole.

The four steps in an AAR are:

\begin{enumerate}
\def\labelenumi{\arabic{enumi}.}
\item
  Review what was supposed to happen (training plans).
\item
  Establish what happened.
\item
  Determine what was right or wrong with what happened.
\item
  Determine how the task should be done differently the next time.
\end{enumerate}

The stated purpose of the AAR is to ``identify strengths and
shortcomings in unit planning, preparation, and execution, and guide
leaders to accept responsibility for shortcomings and produce a fix.''
\end{quote}

We combined the AAR with our paragogy principles --

\begin{enumerate}
\def\labelenumi{\arabic{enumi}.}
\item
  Changing context as a decentered center.
\item
  Meta-learning as a font of knowledge.
\item
  Peers provide feedback that wouldn't be there otherwise.
\item
  Paragogy is distributed and nonlinear.
\item
  Realize the dream if you can, then wake up!
\end{enumerate}

and went through steps 1-4 for each principle to look at how well it was
implemented at P2PU. This process helped generate new policies that
could be pursued further at P2PU or similar institutions. By presenting
our paper at the \href{http://okfn.org/okcon/}{Open Knowledge Conference
(OKCon)}, we were able to meet~P2PU's executive director, Philipp
Schmidt, as well as other highly-involved P2PU participants; our
feedback may ultimately have contributed to shaping the development
trajectory for P2PU.

In addition, we developed a strong prototype for constructive engagement
with peer learning that we and others could deploy again. In other
words, variants on the AAR and the paragogical principles could be
incorporated into future learning contexts as platform features
{{[}3{]}} or re-used in a design/administration/moderation approach
{{[}4{]}}.~ For example, we also used the AAR to help structure our
writing and subsequent work on \href{http://paragogy.net}{paragogy.net}.

\hypertarget{case-study-2-peeragogy-year-one.}{%
\subsection{Case Study 2: Peeragogy, Year
One.}\label{case-study-2-peeragogy-year-one.}}

We surveyed members of the Peeragogy community with questions similar to
those used by Boud and Lee {{[}1{]}} and then identified strengths and
shortcomings, as we did with the AAR above.

\hypertarget{questions}{%
\subsection{Questions}\label{questions}}

These were discussed, refined, and answered on an etherpad: revisions to
the original set of questions, made by contributors, are marked in
italics.

\begin{enumerate}
\def\labelenumi{\arabic{enumi}.}
\item
  Who have you learned with or from in the Peeragogy project? \emph{What
  are you doing to contribute to your peers' learning?}
\item
  How have you been learning during the project?
\item
  Who are your peers in this community, and why?
\item
  What were your expectations of participation in this project?
  \emph{And, specifically, what did you (or do you) hope to learn
  through participation in this project?}
\item
  What actually happened during your participation in this project (so
  far)? \emph{Have you been making progress on your learning goals (if
  any; see previous question) -- or learned anything unexpected, but
  interesting?}
\item
  What is right or wrong with what happened (Alternatively: how would
  you assess the project to date?)
\item
  How might the task be done differently next time? (What's ``missing''
  here that would create a ``next time''\emph{, ``sequel'', or
  ``continuation''?})
\item
  \emph{How would you like to use the Peeragogy handbook?}
\item
  \emph{Finally, how might we change the questions, above, if we wanted
  to apply them in your peeragogical context?}
\end{enumerate}

\hypertarget{reflections-on-participants-answers}{%
\subsection{\texorpdfstring{\textbf{Reflections on participants'
answers}}{Reflections on participants' answers}}\label{reflections-on-participants-answers}}

Some of the tensions highlighted in the answers are as follows:

\begin{enumerate}
\def\labelenumi{\arabic{enumi}.}
\item
  \emph{Slow formation of ``peer'' relationships.} There is a certain
  irony here: we are studying ``peeragogy'' and yet many respondents did
  not feel they were really getting to know one another ``as peers'', at
  least not yet. Those who did have a ``team'' or who knew one another
  from previous experiences, felt more peer-like in those relationships.
  Several remarked that they learned less from other individual
  participants and more from ``the collective'' or ``from everyone''. At
  the same time, some respondents had ambiguous feelings about naming
  individuals in the first question: ``I felt like I was going to leave
  people out and that that means they would get a bad grade - ha!'' One
  criterion for being a peer was to have built something together, so by
  this criterion, it stands to reason that we would only slowly become
  peers through this project.
\item
  \emph{``Co-learning'', ``co-teaching'', ``co-producing''?} One
  respondent wrote: ``I am learning about peeragogy, but I think I'm
  failing {{[}to be{]}} a good peeragogue. I remember that Howard
  {{[}once{]}} told us that the most important thing is that you should
  be responsible not only for your own learning but for your peers'
  learning. {{[}\ldots{}{]}} So the question is, are we learning from
  others by ourselves or are we {{[}\ldots{}{]}} helping others to
  learn?'' Another wrote: ``To my surprise I realized I could contribute
  organizationally with reviews, etc. And that I could provide some
  content around PLNs and group process. Trying to be a catalyst to a
  sense of forward movement and esprit de corps.''
\item
  \emph{Weak structure at the outset, versus a more ``flexible''
  approach.} One respondent wrote: ``I definitely think I do better when
  presented with a framework or scaffold to use for participation or
  content development. {{[}\ldots{}{]}} (But perhaps it is just that I'm
  used to the old way of doing things).'' Yet, the same person wrote:
  ``I am interested in {{[}the{]}} applicability {{[}of peeragogy{]}} to
  new models for entrepreneurship enabling less structured aggregation
  of participants in new undertakings, freed of the requirement or need
  for an entrepreneurial visionary/source/point person/proprietor.''
  There is a sense that some confusion, particularly at the beginning,
  may be typical for peeragogy. With hindsight, one proposed
  ``solution'' would be to ``have had a small group of people as a cadre
  that had met and brainstormed before the first live session
  {{[}\ldots{}{]}} tasked {{[}with{]}} roles {{[}and{]}} on the same
  page''.
\item
  \emph{Technological concerns.} There were quite a variety, perhaps
  mainly to do with the question: how might a (different) platform
  handle the tension between ``conversations'' and ``content
  production''? For example, will Wordpress help us ``bring in'' new
  contributors, or would it be better to use an open wiki? Another
  respondent noted the utility for many readers of a take-away PDF
  version. The site (peeragogy.org) should be ``{{[}a{]}} place for
  people to share, comment, mentor and co-learn together in an ongoing
  fashion.''
\item
  \emph{Sample size.} Note that answers are still trickling in. How
  should we interpret the response rate? Perhaps what matters is that we
  are getting ``enough'' responses to make an analysis. One respondent
  proposed asking questions in a more ongoing fashion, e.g., asking
  people who are leaving: ``What made you want to quit the project?''
\end{enumerate}

\hypertarget{discussion}{%
\subsection{Discussion}\label{discussion}}

\begin{quote}
\textbf{Lisewski and Joyce}: In recent years, the tools, knowledge base
and discourse of the learning technology profession has been bolstered
by the appearance of conceptual paradigms such as the `five stage
e-moderating model' and the new mantra of `communities of practice'.
This paper will argue that, although these frameworks are useful in
informing and guiding learning technology practice, there are inherent
dangers in them becoming too dominant a discourse. {[}5{]}
\end{quote}

Instead of a grand narrative, Peeragogy is a growing collection of case
studies and descriptive patterns.~ As we share our experiences and make
needed adaptations, our techniques for doing peer learning and peer
production become more robust. Based on the experiences described above,
here are a few things people may want to try out in future projects:

\begin{itemize}
\tightlist
\item
  ``Icebreaking'' techniques or a ``buddy system''; continual~
  refactoring into teams.
\item
  Maintain a process diagram that can be used to ``triage'' new ideas
  and effort.
\item
  Prefer the ``good'' to the ``best'', but make improvements at the
  platform level as needed.
\item
  Gathering some information from everyone who joins, and, if possible,
  everyone who leaves.
\end{itemize}

\hypertarget{references}{%
\subsection{References}\label{references}}

\begin{enumerate}
\def\labelenumi{\arabic{enumi}.}
\item
  Boud, D. and Lee, A. (2005).
  \href{http://manainkblog.typepad.com/faultlines/files/BoudLee2005.pdf}{`Peer
  learning' as pedagogic discourse for research education}.
  \emph{Studies in Higher Education}, 30(5):501--516.
\item
  Joseph Corneli and Charles Jeffrey Danoff,
  \href{http://ceur-ws.org/Vol-739/paper_5.pdf}{Paragogy}, in Sebastian
  Hellmann, Philipp Frischmuth, Sören Auer, and Daniel Dietrich (eds.),
  \emph{Proceedings of the 6th Open Knowledge Conference, Berlin,
  Germany, June 30 \& July 1, 2011},
\item
  Joseph Corneli and Alexander Mikroyannidis (2011).
  \href{http://greav.ub.edu/der/index.php/der/article/view/188/330}{Personalised
  and Peer-Supported Learning: The Peer-to-Peer Learning Environment
  (P2PLE)}, \emph{Digital Education Review}, 20.
\item
  Joseph Corneli,
  \href{http://paragogy.net/ParagogicalPraxisPaper}{Paragogical Praxis},
  \emph{E-Learning and Digital Media} (ISSN 2042-7530), Volume 9, Number
  3, 2012
\item
  Lisewski, B., and P. Joyce (2003). Examining the Five Stage
  e-Moderating Model: Designed and Emergent Practice in the Learning
  Technology Profession, \emph{Association for Learning Technology
  Journal}, 11, 55-66.
\end{enumerate}
