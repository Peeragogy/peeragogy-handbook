\paragraph{Definition:} Moderation seems to have a double meaning: there's
moderation as in moderation in all things and moderation as in keeping a
discussion going smoothly. Actually, both of them are about the same
thing.

\paragraph{Problem:}
\href{http://peeragogy.org/organizing-a-learning-context/participation/}{Participation}
in online forums tends to follow a ``power law,'' with vastly unequal
engagement.

\paragraph{Solution:} If you want to counteract this tendency, one
possibility would simply be for the most active participants to step
back, and moderate how much they speak. This is related the the
\href{http://peeragogy.org/patterns-usecases/patterns-and-heuristics/carrying-capacity/}{Carrying
Capacity} pattern and the
\href{http://peeragogy.org/practice/antipatterns/misunderstanding-power/}{Misunderstanding
Power} anti-pattern: check those out before you proceed.

\paragraph{Examples:} Occupy Wall Street used a technique that they called
the ``\href{http://en.wikipedia.org/wiki/Progressive_stack}{progressive
stack}.'' There are lots of other strategies to try.

\begin{quote}
\textbf{The Co-Intelligence Institute}: Why is a fishbowl more
productive than debate? The small group conversations in the fishbowl
tend to de-personalize the issue and reduce the stress level, making
people's statements more cogent. Since people are talking with their
fellow partisans, they get less caught up in wasteful adversarial games.
\end{quote}

\paragraph{Challenges:} In a distributed project, there are many
side-conversations, and it is impossible (and would be undesirable) for
any one person to moderate all of them. The difficulty occurs if one of
these conversations becomes uncomfortable for one or more participants,
for whatever reason. Rather than depending on one central moderator,
it's useful for everyone in the project to be aware of the principles
underlying effective moderation, and apply them together even in small
side-projects.

\paragraph{What's Next:} We recently ran a Paragogical Action Review to
elicit feedback from participants in the Peeragogy project. Some of them
brought up dissatisfactions, and some of them brought up confusion. Can
we find ways to bring these concerns front-and-center, without
embarrassing the people who brought them up?
