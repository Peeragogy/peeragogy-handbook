\subsection{Summary}

\begin{quote}
The metrics for learning in corporations are business metrics based on
financial data. Managers want to know: ``Has the learning experience
enhanced the workers' productivity?''
\end{quote}
\subsection{Follow the money}

When people ask about the ROI of informal learning, ask them how they
measure the ROI of formal learning. Test scores, grades,
self-evaluations, attendance, and certifications prove nothing. The ROI
of any form of learning is the value of changes in behavior divided by
the cost of inducing the change. Like the tree falling over in the
forest with no one to hear it, if there's no change in behavior over the
long haul, no learning took place. ROI is in the mind of the beholder,
in this case, the sponsor of the learning who is going to decide whether
or not to continue investing. Because the figure involves judgment, it's
never going to be accurate to the first decimal place. Fortunately, it
doesn't have to be. Ballpark numbers are solid enough for making
decisions.

\href{http://vimeo.com/45989089}{Assessing Workplace Learning} from
\href{http://vimeo.com/user7021511}{Jay Cross} on
\href{http://vimeo.com}{Vimeo}.

The process begins before the investment is made. What degree of change
will the sponsor accept as worthy of reinvestment? How are we going to
measure that? What's an adequate level of change? What's so low we'll
have to adopt a different approach? How much of the change can we
attribute to learning? You need to gain agreement on these things
beforehand. Monday morning quarterbacking is not credible. It's crazy to
assess learning immediately after it occurs. You can see if people are
taking part or if they're complaining about getting lost, but you cannot
assess what sticks until the forgetting curve has ravaged the learners'
memories for a few months. Without reinforcement, people forget most of
what they learn in short order. It's beguiling to try to correlate the
impact of learning with existing financial metrics like increased
revenues or better customer service scores. Done on its own, this
approach rarely works because learning is but one of many factors that
influence results. Was today's success due to learning or the ad
campaign or weak competition or the sales contest or something else? The
way to assess how people learn is to ask them. How did you figure out
how to do this? Who did you learn this from? How did that change your
behavior? How can we make it better? Too time consuming? Not if you
interview a representative sample. For example, interviewing less than
100 people out of 2000 yields an answer within 10\% nineteen times out
of twenty, a higher confidence level than most estimates in business.
Interviewing 150 people will give you the right estimate 99\% of the
time.
