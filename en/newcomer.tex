\begin{quote}
Unless there is a new person to talk to, a lot of the ``education
stuff'' we do could grow stale. Many of the patterns and use cases for
peeragogy assume that there will be an audience or a new generation of
learners - hence the drive to create a \emph{guide}. Note that the
\emph{newcomer} and the \emph{wrapper} may work together to make the
project accessible. Even in the absence of actual newcomers, we're often
asked to try and look at things with a ``beginner's mind.''
\end{quote}
Joe Corneli's
\href{http://peeragogy.org/practice/heuristics/heartbeat/}{example}
evoked my own experience energing the Peeragogy community. As a
Peeragogy newcomer, I was kindly welcomed and mentored by Joe, Howard,
Fabrizio, and others. I asked naiive questions and was met with patient
answers, guiding questions, and resource links. Concurrently, I
bootstrapped myself into a position to contribute to the workflow by
editing the live manuscript for consistency, style, and continuity. The
concrete act of editing and fact-checking this relatively (to me)
unfamiliar topic in physical isolation rapidly raised my understanding
of the field. I also returned to the
\href{http://socialmediaclassroom.com/host/peeragogy}{Social Media
Classroom} forums to follow up on early offers of editing help from
recently uninvolved particpants, resulting in the rekindled interest of
several (if not an overwhelming army) new editors. - \emph{Charlotte
Pierce (written while editing this page).}

\subsubsection{ADDITIONAL READING}

\href{http://socialmediaclassroom.com/host/peeragogy/forum/suggest-new-discussion-topics\#comment-1796}{Practical
advice}\href{http://socialmediaclassroom.com/host/peeragogy/forum/suggest-new-discussion-topics\#comment-1796}{for
receiving newcomers in Peeragogy}. - \emph{by Regis Barondeau}
