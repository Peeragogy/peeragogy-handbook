Facilitation is a process of helping groups work cooperatively and
effectively. Facilitation can be particularly helpful for individuals
who, based on a certain level of insecurity or inexperience, tend to
lurk rather than participate. At the same time, in peeragogy, a
facilitator isn't necessarily an ``authority.'' Rather, facilitation
work is done in service to the group and the group dialogue and process.
For example, a facilitator may simply ``hold space'' for the group by
setting up a meeting or a regular series of discussions.

\hypertarget{co-facilitating-in-peer-to-peer-learning}{%
\subsection{Co-facilitating in peer-to-peer
learning}\label{co-facilitating-in-peer-to-peer-learning}}

Co-facilitation can be found in collaborations between two or
more~people who need each other to complete a task, for example, learn
about a~given subject, author a technical report, solve a problem, or
conduct~research. Dee Fink writes that ``in this process, there has to
be some kind of~change in the learner. No change, no learning''
{{[}1{]}}. Significant learning~requires that there be some kind of
lasting change that is important in terms of the learner's life; in
peeragogy, one way to measure the~effectiveness of co-facilitation is to
look for a change in the peer~group.

Co-facilitation~roles can be found in groups/teams like basketball,
health, Alcoholics~Anonymous, spiritual groups, etc. For example,
self-help groups are~composed of people who gather to share common
problems and experiences~associated with a particular problem,
condition, illness, or personal~circumstance.~ There are some further
commonalities across different settings.~ Commenting on the work of Carl
Rogers:

\begin{quote}
\textbf{Godfrey Barrett-Lennard}: The educational~situation which most
effectively promotes significant learning is one in which (1) threat to
the self of the learner is reduced~a minimum, and (2) differentiated
perception of the field of experience is facilitated. {{[}2{]}}
\end{quote}

Part of the facilitator's role~is to create a safe place for learning to
take place; but they~should also challenge the participants.

\begin{quote}
\textbf{John Heron}: Too much hierarchical control, and~participants
become passive and dependent or hostile and resistant. They~wane in
self-direction, which is the core of all learning. Too much~cooperative
guidance may degenerate into a subtle kind of nurturing~oppression, and
may deny the group the benefits of totally autonomous~learning. Too much
autonomy for participants and laissez-faire on your~part, and they may
wallow in ignorance, misconception, and chaos. {{[}3{]}}
\end{quote}

\hypertarget{co-facilitating-discussion-forums}{%
\subsection{Co-facilitating discussion
forums}\label{co-facilitating-discussion-forums}}

If~peers are preparing a forum discussion, here are some ideas from
``\href{http://ctb.ku.edu/en/tablecontents/section_1180.aspx}{The
Community Tool Box}'', that can be helpful as guidelines:

\begin{itemize}
\tightlist
\item
  Explain the importance of collaborative group work and make it a
  requirement.
\item
  Establish how you will communicate in the forum.
\item
  Be aware of mutual blind spots in facilitating and observing others.
\item
  Watch out for different rhythms of intervention.
\end{itemize}

\hypertarget{co-facilitating-wiki-workflows}{%
\subsection{Co-facilitating~wiki
workflows}\label{co-facilitating-wiki-workflows}}

A good place to begin for any group of co-facilitators working~with a
wiki are Wikipedia's famous ``5 Pillars.''

\begin{itemize}
\tightlist
\item
  Wikipedia is an encyclopedia.
\item
  Wikipedia writes articles from a neutral point-of-view.
\item
  Wikipedia is free content that anyone can edit, use, modify, and
  distribute.
\item
  Editors should interact with each other in a respectful and civil
  manner.
\item
  Wikipedia does not have firm rules.
\end{itemize}

\hypertarget{co-facilitating-live-sessions}{%
\subsection{Co-facilitating live
sessions}\label{co-facilitating-live-sessions}}

Learning~experiences in live sessions are described in the article
\href{http://dmlcentral.net/blog/howard-rheingold/learning-reimagined-participatory-peer-global-online}{Learning
Re-imagined:~Participatory, Peer, Global, Online} by Howard Rheingold,
and many of these points are revisited in the handbook section on
\href{http://peeragogy.org/real-time-meetings/}{real-time tools}.~ But
we want to emphasize one point here:

\begin{quote}
\textbf{Howard Rheingold}: Remember you came together with your peers~to
accomplish something, not to discuss an agenda or play with
online~tools; keep everything as easily accessible as possible to ensure
you~realize your goals.
\end{quote}

\hypertarget{references}{%
\subsection{References}\label{references}}

\begin{enumerate}
\def\labelenumi{\arabic{enumi}.}
\item
  Fink, L. D (2003). \emph{Creating significant learning experiences: An
  integrated approach to designing college courses}. John Wiley \& Sons.
\item
  Barrett-Lennard, G. T. (1998). \emph{Carl Rogers' Helping System:
  Journey \& Substance}. Sage.
\item
  Heron, J. (1999). \emph{The complete facilitator's handbook}. London:
  Kogan Page.
\end{enumerate}

~
