This section addresses
\href{http://edfutures.net/index.php?title=Practitioner\_Research}{research
practitioners}. At a high level, the questions are:

\begin{itemize}
\item
  How can we understand peer learning better?
\item
  How can we do research ``the peeragogical way''?
\item
  How do we bring research into our peer learning activities?
\end{itemize}
We'll outline three different lines of detailed questioning that expand
on these points. These could be studied in many different ways.

\subsection{Question A. Which activities have the biggest payoff for
learners, in terms of our learning model?}

The preliminary question is, what is the learning model? For example,
our
\href{http://cmapspublic3.ihmc.us/rid=1K81VLSK7-1RL0RQ4-WZK/Peeragogy\%20Cmap.cmap}{concept
map} provides one model of ``peeragogy'' as a subject, but to make this
into a ``learning model'', we would have to do some further work. What
will we accept as evidence of learning or progress?

This is to do with whether we think of learning as something that can
happen conceptually, or only ``in practice''. In the peeragogy project,
we follow the latter view, which is in line with what Peter Sloterdijk
says about learning through direct participation:

\begin{quote}
\emph{The consequences of Foucault's suggestions will only be
appreciated if there is one day a fully worked-out form of General
Disciplinics -- which would probably take a century to develop. Its
implantation would require a suitably contemporary transformation of
universities and colleges, both in the structuring of the so called
`subjects' or `courses' and in the basic assumptions of academic
pedagogy -- which, against its better judgement, still clings to the
briefcase-and-box theory, where teaching and learning is nothing but
transferring knowledge from the professor's briefcase to the students'
file boxes, even though it has long been known that learning can only
take place through a direct participation in the disciplines.
Establishing an academic system with discipline-based content and
methods would at once be the only realistic way to counteract the
atrophy of the educational system, founded on a reformed idea of the
subjects and tasks of a Great House of Knowledge.}{[}1{]}\emph{}
\end{quote}
In general, a discipline will ``come with'' its own learning model and
its own sense of ``progress''. Given that we can get ahold of the
learning model in our discipline of choice, then we can start to address
this first question.

\subsubsection{An hypothesis}

A study plan that puts learners into contact with new concepts and
techniques in such a way that they are not overwhelmed, and yet are
continually challenged will be the best. For example, this could be done
by solving progressively harder problems (and going back to easier ones
when you get stuck).

\subsubsection{An experiment}

Look at different interaction histories and ``add up'' the concepts
learned and the heuristics used. There are some features of social
interaction (like asking questions) that we could use to guess how much
people knew in advance.

\subsection{Question B. Does our instrumentation of the learning model
have reasonable fidelity?}

In the best possible scenario, we have a detailed model of learning that
indicates clearly what people know, and how they got there, where they
can go next, and what steps are required. In practice, the model will
probably be a bit more sloppy.

\subsubsection{An hypothesis}

The quality of the learning model will be determined by the quality of
our underlying representation of ``domain'' or ``disciplinary''
knowledge.

\subsubsection{An experiment}

If we have a computer-based peeragogy platform that can support
``standard'' coursework, and a teacher who is willing to run a course
using this platform, then we can see whether our instrumentation
predicts ``traditional'' measures of success in the course.

\subsection{Question C. Which interventions have the biggest payoff?}

\subsubsection{An hypothesis}

We should be able to use models of learning effects to test out a wide
range of possible interventions.

\subsubsection{An experiment}

Make the given intervention, and measure the total impact on learning
across the population. (This requires a fairly sophisticated learning
model and research apparatus!)

\subsection{Some further reflections}

How you decide to learn, and how you decide to do research, will have
some significant influence on the sort of group you convene! If you plan
to follow a clearly delineated pre-existing course, maybe you don't
``need'' peeragogy. On the other hand, if you're aiming to build peer
support that works, you will definitely want to put some thought into
your learning model!

\subsection{Reference}

\begin{enumerate}
\item
  Sloterdijk, P. (2013). \emph{You Must Change Your Life}, Polity Press.
  (Tr. Wieland Hoban)
\end{enumerate}
