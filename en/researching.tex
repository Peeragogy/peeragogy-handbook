Author: Joe Corneli

If you have a research bent, by this point, you may be asking yourself
questions like these: \emph{How can we understand peer learning better?}
\emph{How can we do research ``the peeragogical way''?} \emph{How do we
combine research and peer learning?}You may also be asking more
technical methodological and instrumentation-level questions: \emph{Do
we have a good way to measure learning?} \emph{Which activities and
interventions have the biggest payoff?} 

This chapter summarizes qualitative
research I did on PlanetMath.org, using the pattern catalog,
as part of my work for my PhD.  In the course of the study, I developed 3
new patterns. 

The first point to make is that although this research was informal,
it is nevertheless (at least in my view) highly rigorous. This is
because the pattern catalog is a relatively stable, socially agreed
upon objct, though it is not fixed for all time.  We can use it to
help identify ``known'' patterns, but we can also extend it with new
patterns -- assuming that we can make an argument to explain why the
new patterns are needed.  The notion of pattern-finding as a process
related to, but distinct from abstraction is described by Richard
Gabriel, who emphasizes that the ``patterns and the social process for
applying them are designed to produce organic order through piecemeal
growth'' ({[}1{]}, p. 31).  

We can use the rigorous-but-informal notion of an expanding pattern
catalog to help address the high-level questions about peeragogical
research mentioned above. The three new patterns I present here are:
Frontend and Backend, Spanning Set, and Minimum Viable Project. These
patterns are both an ``outcome'' of research in a real peer learning
context -- and also a reflection on peeragogical research methods. Like
the other peeragogy patterns, they are tools you can use in your own
work.

\subsection{Study design}

The study was based on interviews with users of a new software system
that we deployed on PlanetMath.org. In the interviews, we covered a wide
range issues, ranging from basic issues of usability all the way to
``deep'' issues about how people think about mathematics. In this
project, I was interested not only in how people collaborate to solve
mathematical problems, but how they think about ``system level'' issues.
The design I had in mind is depicted in the figures below. The key idea
is that patterns emerge as ``paths in the grass'', or ``desire lines''.
The idea that learning design has emergent features is not itself new;
see e.g. {[}2{]}. What's new here is a characterization of the key
patterns for \emph{doing} emergent design in a peer learning context.

\begin{figure}
\begin{center}
\includegraphics[width=.7\textwidth]{../pictures/PeeragogyEDU.jpg}
\end{center}
\caption*{Map of a virtual campus}
\end{figure}

\begin{figure}
\begin{center}
\includegraphics[width=.7\textwidth]{../pictures/PeeragogyEDU-paths.jpg}
\end{center}
\caption*{Peeragogy patterns as loci for ``paths in the grass''}
\end{figure}

\subsection{Initial thematic analysis}

Before describing the new patterns, I will briefly summarize the themes
I identified in the interviews. This can serve as an overview of the
current features and shortcomings of PlanetMath system for people who
are not familiar with it.

\begin{itemize}
\item
  \textbf{``Necessary but not sufficient''.} Users identified a range of
  essential features, like a critical mass of other users to talk to.
\item
  \textbf{``Nice to have''.} It was also easy to identify a bunch of
  cool new ``dream'' features.
\item
  \textbf{Challenges with writing mathematics.} PlanetMath uses \LaTeX,
  which isn't entirely easy to learn (however, we could adapt the
  software to help new users get started).
\item
  \textbf{Progressive problem solving.} The new PlanetMath contains
  problems and solutions, but no easy way to talk about conjectures.
  Users would like a better way to share and discuss work-in-progress.
\item
  \textbf{Personal history, social constructivism.} Better features for
  tracking and, where appropriate, sharing, personal history would help
  users make sense of what's happening in the site.
\item
  \textbf{Regulating learning in a social/mediated context.} Different
  users would look for different things to keep them on track (e.g.
  expert guidance, or a due ``sense of urgency'' in feedback from
  peers).
\item
  \textbf{Comparison with roles in other contexts.} Many users expect a
  ``service delivery'' style that is not entirely consistent with the
  ``open'' production model used in a free/open, volunteer-driven
  project. We need to work more on responsiveness in every aspect of the
  project (keeping in mind that most participants are volunteers).
\item
  \textbf{Concreteness as a criterion of quality.} ``Knowing what you
  can do,'' both with the software and with the content, is important.
  On the content level, pictures help.
\item
  \textbf{Personalization and localization.} The system has a
  practically unlimited potential for personalization, although many
  basic personalized interaction modes have not been built yet.
\end{itemize}
\subsection{Pattern analysis}

At the next level of analysis, the themes extracted above were further
analysed in relationship to the peeragogy pattern catalog.

\subsubsection{Frontend and Backend}

Although mathematics is a relatively formal domain, many of the
motivations for using PlanetMath map onto what Zimmerman and Campillo
call informal problem solving {[}3{]}. Informal problems are are
personally defined and possess openended boundary conditions, i.e., are
situated within an ``open world.'' I like thinking about this in terms
of the way a car works. You can model the steering and drive system with
classical mechanics. But you ultimately need to model the engine with
statistical mechanics and chemistry. You get in a car and start driving
and usually it works more or less the way you'd expect. This is how it
works with other ``formal'' systems. You queue here, sign there, pay
your fee, and it's all done. With informal systems, it's messier. Of
course, the car's engine has a detailed diagram, and for a mechanic,
it's just another ``formal'' system. And, yes, the streets at rush hour
can get very messy. It's all relative. The broader point is that where
ever it appears, ``formal'' is straightforward. In order to design a
collaborative system, you want to bring in enough messiness to let new
and unexpected features emerge -- support for serendipity -- but you
also need to be aware of the user's experience. As another analogy,
imagine a butcher shop. You want the user to be able to take away nice
little packages of meat, you don't want them cutting up whole cows.
Leave that to the pros. The idea of Frontend and Backend is related to
the pattern of the
``\href{http://peeragogy.org/practice/newcomer/}{Newcomer}'' pattern,
since typically one will not expect the user of a system to know how to,
or to be motivated to, work with backend features of a system until they
have mastered at least some of the frontend features. It would be rare
to find an auto mechanic who did not know how to drive. David Cavallo
wrote about an ``engine culture'' in rural Thailand, in which
structurally open systems made some of the ``backend'' features of
internal combustion engines a part of daily life {[}4{]}. In PlanetMath,
we have an ``open engine'', but not necessarily an open engine culture
(users expect a level of service provision). The Frontend and Backend
pattern clearly lends itself to standard service provision, but it can
also be part of paragogical activity. For example, sophisticated and
committed users of the PlanetMath website could focus energy on
supporting individual newcomers, by helping them develop a high-quality
sub-site on their topic of interest. Such effort would simultaneously
inform the development of backend features, and help raise the profile
of the site as a whole. The pattern is in this way associated with
\href{http://peeragogy.org/practice/focusing-on-a-specific-project/}{Focusing
on a Specific Project} and with the Divide pattern.

\subsubsection{Spanning Set}

You may be able to get what you need without digging - but if you do
need to dig, it would be very good to get some indication about which
direction to dig in. At the content level, this might be achieved by
using high-level ``topic articles'' as a map to the content. But there
is another broader interpretation of this pattern that related to but
distinct from Frontend and Backend - we call this the Spanning Set. In
general, the Spanning Set might be made up of people, or media objects.
In a standard course model, there is one central node, the teacher, who
is responsible for all teaching and course communication. In large
online courses, this model can be is scaled up:

\begin{quote}
\textbf{Anonymous study participant}: {[}E{]}veryone's allocated a
course tutor, who might take on just a half-dozen students - so, they're
not the overall person in charge of the course, by any means.
\end{quote}
Another version is the classical master/apprentice system, in which
every apprentice is supervised by a certified master. In the typical
online Q\&A context, these roles are made distributed, and are better
modeled by power laws than by formal gradations. A ``spanning set'' of
peer tutors could help shift the exponent attached to the power law in
massive courses. We can imagine a given discussion group of 100 persons
that is divided according to the so-called
\href{http://www.wikipatterns.com/display/wikipatterns/90-9-1+Theory}{90/9/1
rule}, so that 90 lurk, 9 contribute a little, and 1 creates the
content. This is what one might observe, for example, in a classroom
with a lecture format. We could potentially shift the system by breaking
the group up, so that each of the 9 contributors leads a small group of
10 persons, at which point, chances are good that some of the former
lurkers would be converted into contributors. At a more semantic level,
we can advance the five paragogical principles and their various
analogues as a candidate description of the fundamental categories and
relationships relevant to peer learning. In practice, principles can
only provide the most visible ``frontend'', and an actual spanning set
is comprised of emergent patterns. In PlanetMath, this would arise from
combining several different features, like a ``start menu'' that shows
what can be done with the site, a
\href{http://peeragogy.org/practice/heartbeat/}{Heartbeat} built of
recurring meetings, and topic-level guides to content. (Note: as a
project with an encyclopedic component, PlanetMath itself can be used to
span and organize a significantly larger body of existing material.)

\subsubsection{Minimum Viable Project}

The Minimum Viable Product approach to software development is about
putting something out there to see if the customer bites {[}5{]}.
Another approach, related to the pattern we just discussed, is to make
it clear what people can do with what's there and see if they engage. We
might call this the Minimum Viable Project, an adjunct to the
``\href{http://peeragogy.org/practice/roadmap/}{Roadmap}'' pattern, and
yet another interpretation of
\href{http://peeragogy.org/practice/focusing-on-a-specific-project/}{Focusing
on a Specific Project}. One way to strengthen the PlanetMath project as
a whole would be to focus on support for individual projects. The front
page of the website could be redesigned so that the top-level view of
the site is project focused. Thus, instead of collecting all of the
posts from across the site - or even all of the threads from across the
site - the front page could collect succinct summary information on
recently active projects, and list the number of active posts in each,
after the model of Slashdot stories or StackExchange questions. For
instance, each Mathematics Subject Classification could be designated as
a ``sub-project'', but there could be many other cross-cutting or
smaller-scale projects.

\subsection{Summary}

This chapter has used the approach suggested by Figure 2 to expand the
peeragogy pattern language. It shows that the peeragogy pattern language
provides a ``meta-model'' that can be used to develop emergent order
relative to given boundary conditions. As new structure forms, this
becomes part of the boundary conditions for future iterations. This
method is a suitable form for a theory of peer learning and peer
production in project-based and cross-project collaborations - a tool
for conviviality in the sense of Ivan Illich. In other words, we're all
in the same boat. The things that peer learners need in order to learn
stuff in a peer produced setting are exactly the same things that
designers and system builders need, too. And one concrete way to assess
our collective learning is in terms of the growth and refinement of our
pattern catalog.

\begin{figure}
\begin{center}
\textbf{Frontend and Backend} \\ Principles and features\\[.1in]

\textbf{Minimum Viable Project} \\ A Specific Project, Roadmap, Heartbeat, Divide, Use or Make\\[.1in]

\textbf{Spanning Set}\\ Paths in the grass
\end{center}
\caption*{Peeragogical emergent design: a tool for conviviality}
\end{figure}



\subsection{References}

\begin{enumerate}
\item
  Gabriel, R. (1996). Patterns of Software. Oxford University Press New
  York.
\item
  Luckin, R. (2010). Re-designing learning contexts: technology-rich,
  learner-centred ecologies. Routledge.
\item
  Zimmerman, B. J. \& Campillo, M. (2003). Motivating self-regulated
  problem solvers. In J. Davidson \& R. Sternberg (Eds.), The psychology
  of problem solving (pp. 233-262). Cambridge University Press New York,
  NY.
\item
  Cavallo, D. P. (2000). Technological Fluency and the Art of Motorcycle
  Maintenance: Emergent design of learning environments (Doctoral
  dissertation, Massachusetts Institute of Technology).
\item
  Ries, E. (2011). The Lean Startup: How today's entrepreneurs use
  continuous innovation to create radically successful businesses. Crown
  Pub.
\end{enumerate}
