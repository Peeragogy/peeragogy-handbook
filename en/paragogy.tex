Main Actor: Joe, who is working on a concept map about peer learning.

1. Joe is working on a concept map about peer learning, and notices that
a lot of the patterns that apply to learning would apply to other social
activities.~ In the end, what's so special about ``learning'' or
``education''?~ Why should it be separated out from the rest of what
humans do.

2. Certainly education itself has an economic facet to it: for some
people, it's a job, and for many, it means future employability.

3. Can we really discuss methods for ``doing peeragogy'' without also
rethinking the economic and productive aspects of education?~~ Joe
decides that ``paragogy'' should at least be introduced into the
``peeragogy'' concept map.

\textbf{Footnote} This quote from Askins and Pain, ``Contact zones:
participation, materiality, and the messiness of interaction'' (2011) in
the conclusion of our essay
"\href{http://peeragogy.org/to-peeragogy/}{From Peer Learning to
`Peeragogy'}" suggests a ``paragogical'' approach to research within a
``contact zone''.~ That is, paragogy is research that happens
``alongside''.
