It is very useful to have an up-to-date public roadmap for the project,
a place where it can be discussed and maintained. This helps
\href{http://peeragogy.org/practice/heuristics/newcomer/}{newcomers} see
where they can jump in. It also gives a sense of the accomplishments to
date, and any major challenges that lie ahead. Remember, the Roadmap
exists as an artifact with which to share current, but never complete,
understanding of the space. Never stop learning!

\subsubsection{Examples}

In the Peeragogy project, once the book's outline became fairly mature,
we could use it as a roadmap, by marking the sections that are
``finished'' (at least in draft), marking the sections where editing is
currently taking place, and marking the stubs (possible starting points
for future contributors). After this outline matured into a real
\href{http://peeragogy.org/table-of-contents/}{table of contents}, we
started to look in other directions for ways to build on our successes
to date, and started working on a
\href{http://peeragogy.org/peeragogy-org-roadmap/}{roadmap for further
development of the website and peeragogy project as a whole}.

\subsubsection{And also}

Note that a shared roadmap is very similar to a
\href{http://peeragogy.org/to-peeragogy/personal-learning-plan/}{Personal
Learning Plan}, or ``paragogical profile''. We've made some
\href{http://campus.ftacademy.org/wiki/index.php/Free\_Technology\_Guild\#Learning\_design}{examples}
of these as we got started working on the Free Technology Guild.

There is a certain roadmappiness to ``presentation of self'', and you
can learn to use this well. For instance, when introducing yourself and
your work to other people, you can focus on highlights like these:

\begin{itemize}
\item
  ``What is the message behind what you're doing?''
\item
  ``How do you provide a model others can follow or improve upon?''
\item
  ``How can others get directly involved with your project?''
\end{itemize}
