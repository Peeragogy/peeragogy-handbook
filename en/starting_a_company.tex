\subsubsection{Introduction}

I think that Peeragogy has flavors -- learning for learning sake for
personal ends in a progression toward learning about the world to take
action as a group. The latter gets heavily into Action Research
(Stringer, 2007), which I love and work heavily in. It is research in
cycles, or loops with feedback to try something, measure it, see how it
worked with the real world, then plan the next question and set of
actions. In each cycle, the group is Learning. I look with that lens at
company start-ups as a perpectual action research cycle. I heard Eric
Reis at SXSW talk about the Lean Startup in this mode, including this
direction in how he even wrote the book. Hypothesis, experiment,
feedback, learn, pivot, next hypothesis\ldots{} Is the group in this
peeragogy learning set knowledge or creating new knowledge? Or through
new knowledge making a change in the world? A great spectrum of
alternatives! Here, my scenario about a company I was on the board on
early on:

\subsubsection{Main actors}

\begin{itemize}
\item
  Cycle 1: Nick, an~MBA student, plus a Computer Science PhD, John,~at a
  major~university. John~had created a unique technology for identifying
  video clips and had no~idea what to do with~it. Nick was an
  ex-engineer learning about how to launch new businesses.
\end{itemize}

\begin{itemize}
\item
  Cycle 2: Additional ``learners'' and co-teachers as board members,
  each adding new learning elements and expertise.
\end{itemize}

\begin{itemize}
\item
  Cycle 3+: New learners as investors and clients.
\end{itemize}

\subsubsection{Main success scenario}

\begin{enumerate}
\item
  Nick and John used a new business plan competition as the catalyst and
  structure to experiment with what ideas might be possible to grow this
  idea. They named it Findable (not the real name; the company did
  launch with some interesting success, but we'll come to that later).
  They brought three other MBAs into the initial group, and within the
  confines of a business plan structure, researched the stereotypical
  elements of a business plan -- addressable market, competition,
  expense and revenue projections, etc. They knew nothing of the area,
  and each person did independent research work to provide some primary
  (interview-based) and secondary (existing text) information about
  their hypothesis of what the technology could do for what audience in
  what environment. They worked hard up until the competition deadline,
  and won the business plan competition, gaining \$15,000 in the process
  plus the attention of some VCs on the judging panel. Each person had
  learned a lot about the technology, the creative process of writing
  the business plan, the rituals involved of asking for money, and the
  flaws in their own plan that they found on its creation. They used
  fairly traditional technology tools: email, shared Word and Excel
  files, telephone, search, and a shared file system to store everything
  that they worked on.
\item
  Nick and Fred wanted to move forward with this project. Their next
  hypothesis was that they could launch this in a specific market. They
  first came to the idea, from the learning from the business plan and
  lots of feedback from the VCs, that they could start with the
  advertising market, as they could now identify and ``tag'' any ad that
  they could find on cable or the internet. They got seed capital from
  three interested parties, who become part of their Action Research
  learning team. They realized to launch that they needed more voices on
  their learning team, so they added their first 3 employees to design
  and sell the product. They also added an advisory board, including
  yours truly, assuming they would be working in the advertising market.
  Technologies? Traditional, though they now included all sorts of tech
  development resources. New information into the mix? They had not put
  together great resources to optimize their time learning, and spent a
  lot of energy keeping up with things, information, and opportunities.
  Learning? Some initial users loved their product, but the market size
  was smaller than they thought\ldots{}plus was very entrenched. The
  companies did not see a real pain point that was being solved.
\item
  Cycle 3 -- what the heck do Nick and Fred do with this? This became
  the true learning phase. Different companies and advisors saw
  different needs for their intriguing product set. They spent 4 years
  (!!!) getting pulled this way and that, using the VC money and needing
  more. (This is VERY much the learning path I see in many small tech
  companies.) Technologies? Same stuff. Learning team? Ebbed and flowed
  with new opportunities and people's patience. My expertise was in the
  ``old'' model, so peaceably left the team (but got options!).
\item
  Cycle 4+ -- a major public company ``found'' them through their
  learning cycles, and found that they solved a pain point. They
  invested a sizeable sum into a chunk of the company, and launched
  their product into that solution. This opened a whole other set of
  learning doors.
\item
  Final cycle -- Happily, I cashed out my options. Two major media
  technology companies ended up buying two areas of key technologies in
  2011, much to my own pocketbook's happiness. Nick and Fred had moved
  on earlier, turning the company learning over to specialized managers.
  I need to see what Nick is up to next\ldots{}.
\end{enumerate}

\subsubsection{Thoughts}

\begin{enumerate}
\item
  Many great patterns were tucked into many cycles of this use case,
  often unspoken assumptions in a new business start-up, including
  environment scanning, codifying specialist knowledge, themes,
  modeling, etc. Consensus building -- an interesting element.
\item
  For me, the additional elements~are~(a)~the scaffolding of the
  ``norms'' of cycles (e.g., business plan creation, a competition, a
  launch of a product) help provide ``norming'' frameworks that can help
  groups achieve as well as limit their looking at the structural norms
  as anything but ``required'' and (b) the lens of Action Research
  Cycles from my own POV. Are we setting a hard limit of providing a
  hypothesis in our co-creation, so we know when we are ``done'' and
  what we have to study? Then once that chunk is done (and CELEBRATED)
  that another hypothesis can be investigated, explored, proven, and
  co-created? I believe that having pre-structured points of learning
  achievement, reflection, and celebration can really help in moving
  forward.
\item
  My own brain is rethinking these issues around content creation after
  hearing Eric Reis speak on how he tested his content creation for his
  \emph{New York Times} best-selling book.
\item
  How are we testing this Handbook, other than living through it? :)
\end{enumerate}
