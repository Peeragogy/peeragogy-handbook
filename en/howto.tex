This document is a practical guide to co-learning, a living document
that invites comment and invites readers to join the community of
editors. The document does not have to be read in linear order from
beginning to end. Material about conceptualizing and convening
co-learning -- the stuff that helps with getting started -- is located
toward the top of the table of contents. Material about use cases,
resources, and assessment is located toward the bottom. Hop around if
you'd like.

We've focused on ``hands-on'' techniques, and you'll probably want to
try things out with your own groups and networks as you read.

\begin{itemize}
\item
  If you want a starter syllabus, check out
  “\href{http://peeragogy.org/peeragogy-in-action/}{Peeragogy in
  Action}” in the resources section.
\item
  If you want to delve directly into the research literature, our
  initial literature survey forms the basis of a
  \href{https://en.wikipedia.org/wiki/Peer\_learning}{Wikipedia
  article}, and the book also includes additional
  \href{http://peeragogy.org/recommended-reading/}{recommended
  readings}.
\item
  For something lighter, many pages in the online version of the book
  include short introductory videos, most of them under one minute long.
  You can do a
  \href{http://www.youtube.com/results?search\_query=peeragogy\&sm=3}{search
  on YouTube} to find these and many of our other videos all in one
  place.
\end{itemize}
This is a living document. If you want to join in, just let us know in
our \href{https://plus.google.com/communities/107386162349686249470}{G+
community}. (which also happens to have the name \emph{Peeragogy in
Action}). If you want to test the waters first, feel free to use the
comment thread attached to each page on peeragogy.org to suggest any
changes or additions, and to share a bit about your story. We might
quote you in future versions of the book to help improve the resource
for others, like this:

\textbf{John Glass}: Reading through the handbook, it strikes me that
the users will be fairly sophisticated folks. They will have ample
knowledge of various tech platforms, resources, a fair amount of formal
education, access and ability to use a number of different gadgets. My
dilemma is that I am thinking that Peeragogy, at its most basic, seems
to be about facilitating P2P learning. As such, at its most basic, it
would be about assisting people to work together to learn something (and
for me, learning encompasses virtually all human behavior, with the
possible exception of that controlled by the autonomic nervous system
and even there I am not sure). In other words, I am thinking it is about
helping anyone learn through partnership with others (group \emph{A})
and yet the handbook appears to be geared toward a rather specialized
group of people (group \emph{B}). I guess what I am looking for is
perhaps some clarification on who is the intended audience, \emph{A} or
\emph{B}? As it stands, I am unsure how it could realistically apply to
\emph{A}\ldots{}Thanks. \\

\textbf{Joe Corneli}: I think that the best thing to do is to do this in
dialog. In addition to groups \emph{A} and \emph{B}, we might need a
group \emph{C}, who would mediate between the two. The assignment would
be something like this: ``Use this to structure the class, and if you
get stuck at any point or if you think the resource isn't the right one,
ask me for help, and we'll work on finding other solutions together.''
At the end of the semester, you might have a new and very different book
tailored to this particular ``audience'' (or ``public'' to use Howard's
term)! That would be cool. The current book definitely isn't a
one-size-fits-all -- I'd say it's more like a sewing machine. In fact, I
think group \emph{C} is the real ``public'' for this book -- not
experts, but people who will say: ``How can I use the ideas and the
process here to do something new?'' What people do with it will
definitely depend on the goal: the model might be \emph{Stand and
Deliver} or it might be \emph{Good Will Hunting} or it may be something
very different. Our long-term goal is not to build a 1000 page version
of the handbook, but to serve as a ``hub'' that can help many different
peer learning projects. The first question is: How can we improve the
usability for you? Rather than tackling the whole book all at once, I
would recommend that we start by dialoging about the
"\href{http://peeragogy.org/peeragogy-in-action/}{Peeragogy in Action}"
syllabus at the end. How would we have to tailor that to suit the needs
of your students? With that in mind, another useful starting point might
be our article on
\href{http://peeragogy.org/the-student-authored-syllabus/}{the student
authored syllabus}. Finally, our motto for the book is: ``\emph{This is
a How-To Handbook}.'' We can talk more about anything that's confusing
and get rid of or massively revise anything that's not useful. That's a
super-micro guide to doing peeragogy.

% \href{http://peeragogy.org/how-to-use-this-handbook/guy2/}{\includegraphics{http://peeragogy.org/wp-content/uploads/2012/04/guy2-202x300.jpg}}
