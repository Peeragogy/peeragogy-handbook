\hypertarget{emergent-roadmap}{%
\section{Emergent Roadmap}\label{emergent-roadmap}}

\begin{quote}
This section reprises the ``What's Next'' steps in all the previous
patterns, offering another view on the project \textbf{Roadmap} in its
emergent form.
\end{quote}

\hypertarget{peeragogy}{%
\subsubsection{▶ Peeragogy}\label{peeragogy}}

We intend to revise and extend the patterns and methods of peeragogy to
make it a workable model for learning, inside or outside of
institutions.

\hypertarget{roadmap}{%
\subsubsection{▶ Roadmap}\label{roadmap}}

If we sense that something needs to change about the project, that is a
clue that we might need to record a new pattern, or revise our existing
patterns.

\hypertarget{reduce-reuse-recycle}{%
\subsubsection{▶ Reduce, reuse, recycle}\label{reduce-reuse-recycle}}

We've converted our old pattern catalog from the \emph{Peeragogy
Handbook} into this paper, sharing it with a new community and gaining
new perspectives. Can we repeat that for other things we've made?

\hypertarget{carrying-capacity}{%
\subsubsection{▶ Carrying capacity}\label{carrying-capacity}}

Making it easy and fruitful for others to get involved is one of the
best ways to redistribute the load. This often requires skill
development among those involved; compare the pattern.

\hypertarget{a-specific-project}{%
\subsubsection{▶ A specific project}\label{a-specific-project}}

We need to build specific, tangible ``what's next'' steps and connect
them with concrete action. Use the Scrapbook to organize that process.

\hypertarget{wrapper}{%
\subsubsection{▶ Wrapper}\label{wrapper}}

We have prototyped and deployed a visual ``dashboard'' that people can
use to get involved with the ongoing work in the project. Let's improve
it, and match it with an improved interaction design for peeragogy.org.

\hypertarget{heartbeat}{%
\subsubsection{▶ Heartbeat}\label{heartbeat}}

Identifying and fostering new and new working groups is a task that can
help make the community more robust. This is the time dimension of spin
off projects described in Reduce, reuse, recycle.

\hypertarget{newcomer}{%
\subsubsection{▶ Newcomer}\label{newcomer}}

A more detailed (but non-limiting) ``How to Get Involved'' walk-through
or ``DIY Toolkit'' would be good to develop. We can start by listing
some of the things we're currently learning about.

\hypertarget{scrapbook}{%
\subsubsection{▶ Scrapbook}\label{scrapbook}}

After pruning back our pattern catalog, we want it to grow again: new
patterns are needed. One strategy would be to ``patternize'' the rest of
the \emph{Peeragogy Handbook.}
