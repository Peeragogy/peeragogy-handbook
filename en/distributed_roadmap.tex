\begin{refsection}

This section reprises the ``What's Next'' steps from all of the
previous patterns, offering another view on the Peeragogy project's
\patternname{Roadmap} in a concrete emergent form.
%
This table has been vetted by project participants, who suggested
revisions.  This led to an outline for a new pattern,
`\patternnameext{Onboarding}', a facilitative process that would
complement \patternname{Newcomer} and \patternname{Wrapper}.

We also tagged many of the items in our list of upcoming tasks with the names of
patterns in the pattern catalog.\footnote{\url{https://goo.gl/tcN3q6}}
However, the techniques that we have described in these patterns reach
far beyond our project.  In many cases, the ideas are ancient, or even
primordial (like \patternname{Heartbeat}).  The patterns can be
applied with or without high technology.  The Peeragogy project is one
of
\begin{quote}
 ``[T]ens of thousands of projects in the traditions of world
improvement \'elan -- without any central committee that would have
to, or even could, tell the active what their next operations should
be.'' \cite[p. 402]{sloterdijk2013change}
\end{quote}
When we talk about ``next steps,'' we aim to show what can be
realistically expected from us.
%% ; one can compare, for instance, the
%% pre-Columbian communal working practice of \emph{minga} or
%% \emph{mink'a}.\footnote{\url{https://es.wikipedia.org/wiki/Minka} (in
%%   Spanish).}
And yet, the emergent \patternname{Roadmap} also goes beyond specific
individuals in the project, since they will come and go.

Despite its tendency to longevity, robustness, and
widespread applicability, peeragogy does have some limits. For example,
the benefits to collaboration
notwithstanding, 
``there is no substitute for really hard
thinking on your own'' \cite{atiyah1974research}.
%% \patternname{Peeragogy} is only useful in limited forms at times when
%% sharing your concerns with others brings little advantage.  For
%% instance, there may be cultural or even geographical barriers that
%% make the cost of building trust prohibitive.
Nevertheless, even the most independent knowledge worker reads and
publishes: ``There is no such thing as private property in
language''~\cite[p.~559]{jakobson1971selected}.  The question is
usually how best -- not whether -- to involve others
\cite{coase1937nature,coases-penguin}.  This relates to another
prospective pattern: `\patternnameext{Don't quit your day job}'.

But there is a further issue that presents more of a paradox.  We
can't dictate the behavior of other participants, and we often can't
guess ourselves what's coming up.  A peeragogical
\patternname{Roadmap} should prepare people for the \emph{absence} of
clear step-by-step direction, the \emph{presence} of different
viewpoints and priorities, and the consequent \emph{requirement} to be
reasonably self-directed.  Many features of the peeragogical approach
would be irrelevant to a project that is managed in a top-down
fashion, and that can rely on other coordination mechanisms (like
contracts) to manage work.  Peeragogy is relevant when we must define
or redefine the problems together.


\paragraph{\hyperref[sec:Peeragogy]{Peeragogy}} 
\PeeragogyWN

\paragraph{\hyperref[sec:Roadmap]{Roadmap}} 
\RoadmapWN

\paragraph{\hyperref[sec:Reduce, reuse, recycle]{Reduce, reuse, recycle}}
\ReduceWN

\paragraph{\hyperref[sec:Carrying capacity]{Carrying capacity}} 
\CarryingWN

\paragraph{\hyperref[sec:A specific project]{A specific project}}
\SpecificWN

\paragraph{\hyperref[sec:Wrapper]{Wrapper}}
\WrapperWN

\paragraph{\hyperref[sec:Heartbeat]{Heartbeat}}
\HeartbeatWN

\paragraph{\hyperref[sec:Newcomer]{Newcomer}}
\NewcomerWN

\paragraph{\hyperref[sec:Scrapbook]{Scrapbook}} 
\ScrapbookWN

\subsection*{Summary}

We introduced nine patterns of peeragogy and connected them to
concrete next steps for the Peeragogy project.  In order to
demonstrate generality, we included examples that illustrate how the
patterns manifest in current Wikimedia projects, and how they could
inform the design of a future university rooted in the values and
methods of peer production.
%
% organization, motivation, and quality
The university metaphor need not constrain the future application of
the ideas.  Nevertheless, a project to translate the free\slash
libre\slash open university from metaphor to reality would present the
opportunity for a new study that could test the generality of Hill's
hypothesis on the mobilizing potential of ``the combination of a
familiar goal (e.g., `simply reproduce Encyclopedia Britannica') with
innovative methods (e.g., `anybody can edit anything')''
\cite[p.~13]{mako-thesis}.
We close by reviewing our approach to the organization of learning,
using three dimensions of analysis that have been previously applied to describe
research on peer production \cite{benkler2015peer}.

\paragraph{Organization} 
Managing work on our project with design patterns that are augmented
with a ``What's next'' follow-through step \emph{decentralizes both
  goal setting and execution} \cite{benkler2015peer}, reintegrating
structure in the form of an emergent \patternname{Roadmap}.  Our
methods apply at varied levels of scale and degrees of formality,
inside or outside of institutional frameworks.  If you already write
patterns, you can add a ``What's next'' step to them to try the
approach for yourself.

\paragraph{Motivation}  The future of learning may be
the \emph{Chartes of programming} \cite{alexander1999origins}, but it will have plenty in common with the
bazaar \cite{raymond2001cathedral}.
%
Philipp Schmidt indicates that \emph{learning is at the core of peer
  production communities} \cite{schmidt+commons-based+2009}.  Our
patterns help to explicate the way these communities work, but more
importantly, we hope these patterns will help potentiate a global
culture of collaborative learning, inside and outside of institutions.

\paragraph{Quality} 
``By intervening in real communities, these efforts achieve a level of
external validity that lab-based experiments cannot''
\cite{benkler2015peer}.  The ``What's next'' annotation piloted here
will be helpful to other design pattern authors who aim to use
patterns as part of a research intervention.  Peer production is not guaranteed to
  out-compete proprietary solutions
\cite{benkler2015peer,free-software-better}: its potential for
success will depend on the way the problems are framed,
and our ability to follow through.

%% \begin{wrapfigure}{l}{.52\textwidth}
%% \vspace{-2.8cm}
%% \hspace{-.15cm}\resizebox{.55\textwidth}{!}{
%% \input{images-floating-tikz}
%% }
%% \hspace{.4cm}
%% \vspace{-2.95cm}
%% \caption{Mnemonic \label{mnemonic}}
%% \vspace{-.6cm}
%% \end{wrapfigure}

\printbibliography[heading=subbibliography]

\begin{figure}
{\centering
\input{images-floating-tikz}

\par}
\vspace{-.5cm}
\caption{Mnemonic \label{mnemonic}}
\end{figure}

\end{refsection}
