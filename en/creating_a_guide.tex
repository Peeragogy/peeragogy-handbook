Meaning-carrying tools, like handbooks or maps, can help people use an
idea. In particular, when the idea or system is only ``newly
discovered'', the associated meanings may not be well understood (indeed
they may not have been created). In such a case, the process of creating
the guide can go hand-in-hand with figuring out how the system works.
Thus, techniques of \href{http://knowledgecartography.org/}{knowledge
cartography} and
\href{http://www.hitl.washington.edu/publications/r-97-47/two.html}{meaning
making} are useful for would-be guide creators. Even so, it is worth
noting that ``the map is not the territory,'' and map-making is only one
facet of shared human activity. Collaboratively refining a pattern is
itself an example of ``Creating a Guide'' - that is, a pattern
description can be thought of as a ``micro-map'' of a specific activity.
