\textbf{Welcome to the Peeragogy Handbook!}

This book, and accompanying website, is a resource for self-organizing
self-learners.

With YouTube, Wikipedia, search engines, free chatrooms, blogs, wikis,
and video communication, today's
\href{http://en.wikipedia.org/wiki/Self-taught\_learner}{self-learners}
have power never dreamed-of before. What does any group of self-learners
need to know in order to self-organize learning about any topic? The
Peeragogy Handbook is a volunteer-created and maintained resource for
bootstrapping peer learning.

This project seeks to empower the worldwide population of self-motivated
learners who use digital media to connect with each other, to
co-construct knowledge of how to co-learn. Co-learning is ancient; the
capacity for learning by imitation and more, to teach others what we
know, is the essence of human culture. We are human because we learn
together. Today, however, the advent of digital production media and
distribution/communication networks has raised the power and potential
of co-learning to a new level.

If you want to learn how to fix a pipe, solve a partial differential
equation, write software, you are seconds away from know-how via
YouTube, Wikipedia and search engines. Access to technology and access
to knowledge, however, isn't enough. Learning is a social, active, and
ongoing process.

\begin{quote}
\emph{What does a motivated group of self-learners need to know to agree
on a subject or skill, find and qualify the best learning resources
about that topic, select and use appropriate communication media to
co-learn it?~ In particular, what do they need to know about peer
learning?}
\end{quote}

This handbook is intended to answer these questions, and in the process,
build a toolbox for co-learning.

Our experience within this project has been that flattened hierarchies
do not necessarily mean decisions go by consensus. The handbook is in
part a collaboration? and in part a collection of single-author works.
Often the lines and voices are blurred. One constant throughout the book
is our interest in making something useful. To this end, the book comes
with numerous~activities, and is available under non-restrictive legal
terms (you can reuse portions of it however you see fit it has been
given a \href{http://creativecommons.org/publicdomain/zero/1.0/}{CC Zero
1.0 Universal Public Domain Dedication}). For those who seek more
evidence-based, scholarly scaffolding for learning practices, we also
maintain a
\href{http://peeragogy.org/resources/literature-review-peeragogy/}{literature
review} of learning theories that pertain to self-organized peer
learning.

Finally, we also include instructions on
\href{http://peeragogy.org/resources/how-to-get-involved/}{how to join
us in further developing this resource}.

Sincerely, \href{http://peeragogy.org/resources/meet-the-team/}{The
Peeragogy Team}
