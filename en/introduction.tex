\begin{quote}
We live where no one knows the answer and the struggle is to figure out
the question. {[}1{]}
\end{quote}

Welcome to the \emph{Peeragogy Handbook}! We want to kick things off
with a candid confession: we're not going to pretend that this book is
perfect. In fact, it's not an ordinary book at all. The adventure starts
when you get out your pen or pencil, and begin marking it up. And it
gets kicked into high gear when you become part of
\emph{Peeragogy in Action}.
If everything goes as planned you'll get a lot of friendly support as
you write, draw, or dance your own peeragogical adventure. But first,
what is \emph{peeragogy}?

Peeragogy is a flexible framework of techniques for peer learning and
peer knowledge production. Whereas pedagogy deals with the transmission
of knowledge from teachers to students, peeragogy is what people use to
produce and apply knowledge together. The strength of peeragogy is its
flexibility and scalability. The learning mind-set and strategies that
we are uncovering in the Peeragogy project can be applied in classrooms,
hackerspaces, corporations, wikis, and interconnected collaborations
across an entire society.

The \emph{Peeragogy Handbook} is a compendium of know how for any group
of people who want to co-learn any subject together, when none of them
is an expert in the particular subject matter -- learning together
without one traditional teacher, especially using the tools and
knowledge available online. What we say in the \emph{Handbook} draws
extensively on our experiences working together on the \emph{Handbook}
-- and our experiences in other collaborative projects that drew us here
in the first place. The best way to learn about peeragogy is to get
involved in putting peeragogy into action. Towards that end, coauthors
and fans of the \emph{Handbook} have an active Google+ community,
conveniently called ``Peeragogy in Action''. We maintain a regular
schedule of weekly meetings that you're welcome to jump into. The
\emph{Handbook} includes a short syllabus, also called ``Peeragogy in
Action'', that you can work through with your own group.

You're warmly invited to combine your local projects with the global
effort, and get involved in making the next edition of the book. That
doesn't necessarily require you to do extensive writing or editing.
We're always interested in new use cases, tricky problems, and
interesting questions. In fact, our view is that any question is a good
question.

Here is a more detailed list of ways in which the current edition of the
\emph{Handbook} is not perfect. You're welcome to add to the list! These
are places where you can jump in and get involved. We believe that
airing our dirty laundry up front will give you a good idea of the
issues and challenges we face putting peeragogy into action. If you're
not intrigued by this sort of challenge, you may be best served by a
different adventure.

\section*{Maintaining a list of useful
resources}\label{maintaining-a-list-of-useful-resources}

We include references and recommended reading in the \emph{Handbook},
and there are a lot more links that have been shared in the \emph{Peeragogy in
Action} community. It's a ongoing task to catalog and improve these
resources -- including books, videos, images, projects, technology, etc.
In short, let's ``Reduce, Reuse, Recycle''! As a good start, Charlotte
Pierce has been maintaining a spreadsheet under the heading ``survey''
in our Google Drive.

\section*{Developing a really accessible DIY
tool-kit}\label{developing-a-really-accessible-diy-tool-kit}

A short ``workbook'' containing interviews and some activities follows
this introduction, but it could be much more interactive. Amanda Lyons
and Paola Ricuarte made several new exercises and drawings that we could
include. A more developed workbook could be split off from the handbook
into a separate publication. It would be great to have something simple
for onramping. For example, the workbook could be accompanied by video
tutorials for new contributors.

Paola Ricuarte points out that a really useful book will be easy to
sell. For teachers interested in peeragogy, this needs to be something
that can be use in workshops or on their own, to write in, to think
through issues. We're partway there, but to improve things, we really
need a better set of activities.

The next time Paola or someone else uses the handbook or handbook to run
a workshop, she can say, ``turn to this page, let's answer this
question, you have 5 minutes.'' There are lots of places where the
writing in the handbook could be made more interactive. One technique
Paola and Amanda used was turning ``statements'' from the handbook into
``questions.''

\section*{Crafting a visual identity}\label{crafting-a-visual-identity}

Amanda also put together the latest cover art, with some collaboration
from Charlotte using inDesign. A more large-scale visual design would be
a good goal for the 4th Edition of the book. Fabrizio Terzi, who made
the handbook cover art for the 1st Edition, has recently been working on
making our website more friendly. So, again, work is in progress but we
could use your help.

\section*{Workflow for the 4th
edition}\label{workflow-for-the-4th-edition}

We've uploaded the content of the book to Github and are editing the
``live'' version of the site in Markdown. For previous print editions,
we've converted to LaTeX. There are a number of workflow bottlenecks:
First, people need to be comfortable updating the content on the site.
Second, it would be good to have more people involved with the technical
editing work that goes into compiling for print. Remember, when we
produce an actual physical handbook, we can sell it. In fact, because
all co-authors have transferred their copyright in this book to the
Public Domain, \emph{anyone} can print and sell copies, convert the
material into new interactive forms, or do just about anything with it.

\section*{Translations}\label{translations}

Translating a book that's continually being revised is pretty much a
nightmare. With due respect to the valiant volunteer efforts that have
been undertaken, it might be more convenient for everyone involved to
just pay professional translators. But for that, we'll need a pretty
serious budget. See below.

\section*{Next steps? What's the future of the
project?}\label{next-steps-whats-the-future-of-the-project}

In short: If we make the Handbook even more useful, then it will be no
problem to sell more copies of it. That is one way to make money to
cover future expenses. It's a paradigmatic example for other business
models we might use in the future. But even more important than
a business model is a sense of our shared vision, which is why
we're working on a ``Peeragogy Creed'' (after the Taekwondo creed, which
exists in various forms, one example is {[}2{]}).

\subsection*{References}\label{references}

\begin{enumerate}
\def\labelenumi{\arabic{enumi}.}
\item
  Joshua Schimel, 2012. ``Writing Science'', Oxford University Press.
\item
  Taekwondo Student Creed, World Martial Arts Academy, \url{http://www.worldtaekwondo.com/handbook.htm}
\end{enumerate}
