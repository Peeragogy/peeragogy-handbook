\begin{quote}
\textbf{Félix Guattari}: \emph{Imagine a fenced field in which there are
horses wearing adjustable blinkers, and let's say that the ``coefficient
of transversality'' will be precisely the adjustment of the blinkers. If
the horses are completely blind, a certain kind of traumatic encounter
will be produced. As soon as the blinkers are opened, one can imagine
that the horses will move about in a more harmonious way.}
(\href{http://nine.fibreculturejournal.org/}{Quoted by Andrew Murphy},
himself quoting Gary Genosko)
\end{quote}

\begin{center}
\includegraphics[width=.85\textwidth]{../pictures/horse.jpg}
\end{center}

From a design point of view: we should be conscious of interfaces that
are ``too loud'', and think about how that is compensated for by
isolation of various forms. With a too-narrow focus, people end up
bumping into each other uncomfortably. However, with an over-wide focus,
things are chaotic in other ways (see
\href{http://peeragogy.org/practice/antipatterns/co-learning-messy-with-lurkers/}{Co-Learning:
Messy with Lurkers}), motivating a narrowing of focus. An effort that
isolates itself will not have the occasion to draw on other resources.

This sometimes goes by the name \emph{Not Invented Here}. But focus is
really only a problem when it becomes overfocus, resulting in
uncomfortable bumps. When that happens, it seems like a good reason to
try to clarify how to engage in a more fruitful manner. Learning how to
manage the uncertainty that comes with experimentation is part of what
makes the postmodern organization tick! (See also:
\href{http://peeragogy.org/antipatterns/navel-gazing/}{Participatory
Design vs Navel Gazing}.)
