\begin{wrapfigure}[]{l}{1.8in}
\includegraphics[width=.4\textwidth]{../pictures/allowed-list.jpg}
\vspace{-50pt}
\end{wrapfigure} Here our aim is to develop the productive ``paragogical'' side of
peeragogy through a discussion of the strategies, joys, and sorrows of
co-working. It complements
the~\href{http://socialmediaclassroom.com/host/peeragogy/freelinking/Co-Working}{co-facilitation}
page.

These questions could apply to our working group(s) here, and to pretty
much any working group in existence:

\begin{itemize}
\item
  How do you pass the ball?
\item
  How do you keep the energy going?
\item
  How do you diagnose where the group is going and make things
  ``intentional'' instead of assumed?
\end{itemize}

And how do we do all of this in a way that takes learning into account?
(Thee proposed ``allowed list'' comes from Simon Sinek, by way of
Fabrizo Terzi and the FTG.)

\subsection{Co-working as the flip side of convening}

Linus Torvalds, interviewed by Steven Vaughan-Nichols for a
Hewlett-Packard publication, had this to say about software development:

\begin{quote}
\emph{The first mistake is thinking that you can throw things out there
and ask people to help. That's not how it works. You make it public, and
then you assume that you'll have to do all the work, and ask people to
come up with suggestions of what you should do, not what they should do.
Maybe they'll start helping eventually, but you should start off with
the assumption that you're going to be the one maintaining it and ready
to do all the work. The other thing--and it's kind of related--that
people seem to get wrong is to think that the code they write is what
matters. No, even if you wrote 100\% of the code, and even if you are
the best programmer in the world and will never need any help with the
project at all, the thing that really matters is the users of the code.
The code itself is unimportant; the project is only as useful as people
actually find it.}
\end{quote}

It is important to understand your users -- and remember that
contributors are a special class of ``user'' with a real time investment
in the way the project works. We typically cannot ``Tom Sawyer''
ourselves into leisure or ease just because we manage to work
collaboratively, or just because we have found people with some common
interests.

The truth is probably somewhere in between Torvalds and Twain. Many
people actively want to contribute! For example, on ``Wikipedia, the
encyclopedia anyone can edit'' (as of
2011)~\href{http://\%20http://www.readwriteweb.com/archives/wikipedias\_goal\_1\_billion\_monthly\_visitors\_by\_2015.php}{as
many as}~80,000 visitors make 5 or more edits per month. This is
interesting to compare with the
\href{http://www.aaronsw.com/weblog/whowriteswikipedia}{fact} that (as
of 2006) ``over 50\% of all the edits are done by just .7\% of the
users\ldots{} 24 people\ldots{}and in fact the most active 2\%, which is
1400 people, have done 73.4\% of all the edits.''~ Similar numbers apply
to other peer production communities.

\begin{center}
\includegraphics[width=.9\textwidth]{../pictures/tom-sawyer.jpeg}
\end{center}

\subsection{A little theory}

In many natural systems, things are not distributed equally, and it is
not atypical for e.g. 20\% of the population to control 80\% of the
wealth (or, as we saw, for 2\% of the users to do nearly 80\% of the
edits). Many, many systems work like this, so maybe there's a good
reason for it.

Let's think about it in terms of ``coordination'' as thought of by the
late Elinor Ostrom. She talked about ``local solutions for local
problems''. By definition, such geographically-based coordination
requires close proximity. What does ``close'' mean? If we think about
homogeneous space, it just means that we draw a circle (or sphere)
around where we are, and the radius of this circle (resp. sphere) is
small. An interesting
\href{http://en.wikipedia.org/wiki/N-sphere\#Volume\_and\_surface\_area}{mathematical
fact} is that as the dimension grows, the volume of the sphere gets
``thinner'', so the radius must increase to capture the same
\emph{d}-dimensional volume when \emph{d} grows! Based on this, we might
guess that the more dimensions a problem has, the more resources we will
need to solve it. From another perspective, the more different factors
impact a given issue, in some sense, the less likely there are to be
small scale, self-contained, ``local problems'' in the first place.

If we think about networks instead of homogeneous space, and notice that
some nodes in the network have more connections than others, then we see
the same issue applies to these nodes: they have more complexity in
their immediate region than the others. This might suggest that such
``central nodes'' (e.g. popular films, popular words, popular websites,
popular people) would, by definition, be less discriminating in terms of
who/what they couple with. On a certain level (weak ties) this is
probably true. But on another level (strong ties) I think it must not be
true -- you can't really have it both ways.

Asking for organizations to work on the ``local'' level of strong ties
when they are ``really'' all about many low-bandwidth weak ties isn't
likely to work well. Google is happy to serve everyone's web requests --
but they can't have just anyone walking in off the street and connecting
devices their network in Mountain View. (Aside: the 2006 article on
Wikipedia quoted above was written by Aaron Swartz, who achieved some
\href{http://www.wired.com/threatlevel/2011/07/swartz-arrest/}{notoriety}
for doing essentially just that, though in his case, it was MIT's
network, not Google's.) We might guess that the more institutionally
committed someone is, the less likely they are to be able to form deep
connections with anyone who is not an integral part of their
institution.

Of course, we don't ``give up''. We aspire to create systems that have
both aspects, systems where a ``dedicated individual can rise to the top
through dint of effort'', etc. These systems are well articulated,
almost like natural languages, which are so expressive and adaptive that
``most sentences have never been said before''. In other words, a
well-articulated system does lend itself to ``local solutions to local
problems'' -- but only because all words are NOT created equal.

\begin{quote}
\emph{My brothers read a little bit. Little words like `If' and `It.' My
father can read big words, too, Like CONSTANTINOPLE and TIMBUKTU.}
\end{quote}

\subsection{Co-working: what is an institution?}

We could talk in this section about Coase's theory of the firm, and
Benkler's theory of ``Coase's Penguin''. We might continue
\href{http://www.aaronsw.com/weblog/perfectinstitutions}{quoting} from
Aaron Swartz. But we will not get so deep into that here: you can
explore it on your own!\footnote{This article was written shortly before Aaron Swarz's untimely death.}
