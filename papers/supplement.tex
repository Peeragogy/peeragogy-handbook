\begin{cframed}[scarlet]
\section{Supplement: The paper we didn't write}

Invited authors had the following proposals for the paper:

\begin{quote}
I'd like to contribute trying to combine and clarify more points in a
visual way for ``research paper'' \& ``research presentation''
submissions.
\end{quote}

\begin{quote}
I'd like to contribute to structuring the argument (e.g. mindmapping)
and making this into a proper, publishable, research paper
(e.g. writing, editing, reference checking). I think it would be
useful for me to re-use the ideas here in the ``Discussion'' section
of my thesis.
\end{quote}

\begin{quote}
``Uses and impact of wikis and other open resources, tools, and
  practices in fields and application areas.''
\end{quote}

\begin{quote}
Can contribute my research into the literature, most of which is
listed here. Also, SlideSpeech interactivity has been put into
production so delivery of the paper (the presentation of it) could be
collaboratively authored in the SlideSpeech system. Here is the
SlideSpeech interactivity tutorial:
\end{quote}

\begin{quote}
I am interested in these areas: Innovative development and/or implementation of wiki applications, Building open systems and tools, Open knowledge and information production, Uses and impact of wikis and other open resources, tools, and practices in fields and application areas \ldots and I will commit to coming up with a cohesive contribution that covers at least some of this.
\end{quote}

\begin{quote}
Interested in Open Ed, Wikis, Sharing OER, lately been using GitHub to remix and share seconday school syllabi \& lesson plans ... interested in how peeragogical patterns might make my efforts more effective \& efficient
\end{quote}

\begin{quote}
I'm interested in the social and cultural aspects of open collaboration: social and cultural capitals (Bourdieu): language, digital competences, transnational networks, and so on. Is there anyone else interested in these topics?
\end{quote}

\subsection{Overview of Contributed Ideas}

Given the digital medium is actually going to look nothing like our traditional concept of space ...

If we're imagining a ``virtual college'' that instantiates the new invisible college of distributed networks, we might begin with a familiar image above like this one:

Figure 1. An abstraction of the familiar college campus

Once we understand the virtual versions of these fixtures of the campus, we can start to look for the ``paths in the grass'' -- those emergent patterns that show how people want to use the space. We can then go beyond this picture and ask how it is wrong or misleading, when we consider the nature of technologies like wikis, and the role they can play in coordinating learning and research in a globally-connected environment.

On pp. 4-5 of The New Invisible College, Caroline Wagner introduces ``five forces'' that match a familiar schema. Networks become the laboratory. The circulation of people and ideas gives a new meaning to the forum. Access to the inevitably ``sticky'' features of geography and rare artifacts generalize the library. Distributed teams working in both local communities and with global interconnection provide a space for informal collaboration, socialization, and learning. But the key feature is emergence -- what happens when we put these other features together.

Just starting with the titles or main ideas from the contributed one-page papers for now, please reorder, add detail, and bridges or context as needed -- and move points into the more FORMAL paper outline below. Each of the top-level headings should probably have 3 or 4 main sub-points -- that's a, b, c, d in the format that Google Docs uses :-). Please also refer to the outline of the Peeragogy Handbook 2.0 -- maybe this paper can be a ``warm-up'' for that. We can also draw writing directly from the Handbook 1.1 if we need/want.

        What is the one idea that connects our various contributions and thoughts...? The thread that will connect everything together... the thesis statement, as it were?

    Designs for a distributed university

        New Invisible College: becoming concrete through self-reflection of research on wikis for research (and education): I can provide a short ``book report'' on Wagner's monograph...

        I want to talk about wikis and wiki-like systems as places for capturing the ``externalities'' from other value-creation systems -- a way to document ``the paths in the grass'' that come from unexpected links between different things

        I'm trying to think about ways to do a tiny (10 day) study on PlanetMath along the lines of the themes I just mentioned...

    Creative confrontation for co-leading

        This seems connected to the Forum that in the Notes section that accompanies my write-up -- it also seems that with the various ``contradictory'' ends, the proposal from Fabrizio moves out of the institution of the forum, and into the ``no-man's land'' of OER communities. As such, it gives a nice statement of ``the problem to be solved'' in our research collaboration -- maybe the ``grand challenge''.

\subsection{Eliminating 90/9/1 and More Possibilities}

    Benefits of wikis in learning

        Stats and charts drawn from the first year of the Peeragogy project in the SMC -- but what stats and charts? What do we want to show or learn about the project from a quantitative point of view? What can we do to connect this to qualitative matters drawn from our own experience with the project?

    An experiment in the efficiency of learning

        Power laws...? Note that the Peeragogy project will also exhibit some power-law like features (no doubt). What can we learn from this? When I think about power laws, I often ask ``what's the exponent?''. What makes the difference between rapid fall off and less-rapid fall off? How does this relate to ``learning''?

*** John mentioned something about learning/hour metric\ldots can we incorporate something like that into our assessment??

        What is our ``contribution to knowledge'' in this paper? (High-level summary of the key findings and ideas.)
% \cite{Peeragogy-2} \cite{College} \cite{peer} \cite{GroupInformatics} \cite{Why} \cite{PeeragogyinAction}
\end{cframed}
