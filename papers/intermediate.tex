\title{Designing Sustainable Learning Communities}

\maketitle
\begin{abstract}

Following a year of productive learning and work culminating in the first edition of The Peeragogy Handbook we reflect here on lessons learned and patterns uncovered. In the second half of the paper we outline our goal: to transition from an innovative theoretical project to a sustainable, easily replicable peer project problem solving accelerator.

\end{abstract}


%%%%%%%%%%%%%%%%%%%%%%%%%%%%%%%%%%%%%%%%%%%%%%%%%%%%%%%%%%%%%%%%%%%%%%%%%%%%%%%%%%%%%% Cut here

\section{The Peeragogy Project}

The Peeragogy Project is a group of adult learners trying to uncover
the most effective ways to do collaborative learning.  We have been
working together in an open online community since January, 2012.  

Our methodology is based on examining and recording how we learn and
work together, and on existing examples and theories of peer learning.
We encourage participants to building and share experiments and case
studies, e.g. the classroom intervention ``5PH1NX'', the virtual
development collaboration hosted by the Free Technology Guild, the new
Bergamo HUB which is intended to be a civic center and maker space.
We have encouraged sharing and critically reflecting on projects like these, with the aim of understanding of the common principles and features.  In this way, the peeragogy project has increasingly come to serve as a distributed, research-based, project incubator.

This paper describes the qualitative aspects of participation in this
project.  Our findings can be used by other projects with a
collaborative focus, and serve as a ``humanities-friendly'' accompaniment to
previous writing about collaboration in open source software
(\cite{OpenAdvice}, \cite{crowstonXdefiningX2003}).  We have a learning focus that is missing from other more general discussions of online community \cite{bacon2012art}.

One of the key features we will attempt to convey in this paper is the multiplicity of possible narratives.  That is to say, each participant in the project has a different view of what it's about, how it is useful for them, and \emph{how} useful it is for them, and this is also true for the authors of the current paper.  The motivations reported by participants in the Peeragogy project include:
\begin{itemize}
\item Acquisition of training or support in a topic or field;
\item Building relationships with interesting people;
\item Finding professional opportunities through other participants;
\item Creating or bolstering a personal network;
\item More organized and rational thinking through dialog and debate;
\item Feedback about their own performance and understanding of the
  topic.
\end{itemize}

Each of those motivations can affect the vitality of the peeragogical process and the end result for the individual participant.  Person-specific motivations are also met by person-specific obstacles. In addition to reports from our own experiences with the project, this paper draws on qualitative survey data to brings in the voices and views of currently other authors, readers, and discussants ($N=250$, $RR=17\%$).

\section{Acknowledgements}

The Peeragogy project was instigated by Howard Rheingold, adapting the Corneli and Danoff \cite{paragogy} work on Paragogy to build a
less-academic, more practical, DIY tract by assembling a group of contributors that included, in addition to the current authors: Bryan Alexander, Paul Allison, Doug Breitbart, Suz Burroughs, Jay Cross, Julian Elve, Mar\'ia Fernanda, James Folkestad, Kathy Gill, Gigi Johnson, Anna Keune, Roland Legrand, Christopher Neal, Ted Newcomb, Stephanie Parker, David Preston, Paola Ricaurte, Stephanie Schipper, and Geoff Walker.

\begin{figure}
\begin{center}
\includegraphics[width=.5\textwidth]{OpenBook1.jpg}
\vspace{-.7in}
\end{center}
\caption{Work by Amanda Lyons on ``Peeragogy in Action''
  \cite{PeeragogyinAction} \label{amanda}}
\end{figure}

\section{Methodology}

The paper is essentially an experience report supplemented by qualitative survey data. These methods seem suitable for phenomenological research that focuses on the question : ``what is it like to participate in the Peeragogy project?''. For Patton (2002), to gather phenomenological data: 
\begin{quote}
"one must undertake in-depth interiews with people who have direcly experienced the phenomenon of interest; that is, they have "lived experience" as opposed to secondhand experience"
\end{quote}
Some of these reflections will generalize to other peer learning projects, and Section \ref{accelerator} provides examples.

We have drawn on the ``research through writing'' method pioneered by Tomlinson \emph{et al.} \cite{tomlinson2012massively}, with reference to earlier work by Christopher Frayling on ``research through design'' \cite{frayling1993research}.  There is also significant design aspect to the project -- effectively ``design through practice''.  Indeed, as Frayling remarked: ``Research is a practice, writing is a practice, doing science is a practice, doing design is a practice, making art is a practice.''

At the core of both this paper (Section \ref{patterns}) and the Peeragogy Handbook (Chapter III) are a set of Alexandrian  patterns that describe the practice of peeragogy.  These patterns developed through observation of the Peeragogy project from its initial days, through discussion and debate among the current authors.

\section{The Peeragogy Handbook}

Peeragogy is a set of replicable techniques and patterns for effective
peer learning and problem solving.  Evidence that this can work comes
from our project: without the support of a unifying formal
institution, over the course of one year, a group of volunteers came
together and wrote a cohesive treatise, the Peeragogy Handbook, that is far richer
than what any one of them might have produced working alone.  Working on the project also had a positive effect on individual projects.

We utilized a variety of technologies to get the job done: live
meetings and a loose bureaucracy of teams and team leaders for
different sub-projects.  We elected to use the Creative Commons Zero
Public Domain Dedication for all of our work, to maximize the
potential for re-use.  Contributors submitted texts, images, and
videos, and in addition to our handbook, we prepared several external
presentations and publications (cf. Figure \ref{amanda}).

We used a blend of synchronous and asynchronous collaboration, and a
whole suite of tools: the Social Media Classroom (a Drupal platform), Diigo, Google Docs,Google+, WordPress, LaTeX, Lulu, Twitter, Blackboard Collaborate, Git,Google Hangouts, Mumble, and more.  

Each of these tools has its own pros and cons and fits a certain
purpose -- peeragogy is not bound to particular digital tools nor the
digital world at all.  For example, a physical interpretation was
given to the project by Anna, Paola, Gigi and Amanda who held a
workshop at the Open Knowledge Conference 2012 in Helsinki\footnote{
  \url{http://okfestival.org/peeragogy-handbook-workshop/} and
  \url{http://www.youtube.com/watch?v=P2UJrN58MVI}, leading to the
  peeragogy contribution to the Open Book \cite{PeeragogyinAction}.},
and again with the work-in-progress by Fabrizio to build a peer learning focused community center.\footnote{\url{http://paragogy.net/Bergamo_HUB:_il_potere_di_innovare_attraverso_la_collaborazione}}

Intuitively, there are bound to be difficulties for a group of peers studying a subject together, outside a traditional classroom or without a teacher. Indeed, peer learning is different from other forms of group effort, the proverbial ``barnraising'' for example, in which the persons involved can be presumed to know how to build barns - or at least to know someone who knows, and stand ready to take orders.  Typically, peers are not experts in learning, didactics, or in the subject they are studying, and are faced with multiple difficulties associated with putting together knowledge about the subject, assembling a suitable learning strategy, and communicating with one another \cite{paragogy}.  In addition to these \emph{a priori} barriers, we needed to create achievable common goals and overcome differences in timezone, English fluency, experience with technology, writing, learning, and engagement styles.

Our strategy was roughly to use each participant's difference as
points of strength.  In essence, we have high cognitive diversity,
and, since we all roughly agree about what we're aiming for in the
project and have allowed ourselves to be inspired by one another, far
less ``identity diversity'', in the sense that ``the more different we
are, the less we agree on what we would like to do''
\cite{Page2008difference}.  We've taken a similar ``open'' approach to
projects and tools, for instance, we've experimented with translation
on the MediaWiki powered Wikibooks platform, project organization with
the Federated Wiki and Trello.  The project has three main ``hubs'':
the Social Media Classroom, where we started writing, the Wordpress
site peeragogy.org, where we host the master copy of the handbook, and
our Google+ community.  Each new tool and contribution poses questions
about how to connect back to the overall flow of work and the book
outline.  It's hard to know in advance which experiments will work!

At a high level, the goal of producing a handbook explaining how other
peers could learn effectively with other peers on a given project kept
things cohesive, and gave us a standard of quality.  Producing the
book proved to be a viable method for researching and deepening our
understanding of peer learning.

Version 1.0 of the Peeragogy Handbook presents a range of techniques
that self-motivated learners can use to connect with each other and
develop stronger communities and collaborations.  We are continuing to
refine the book in the open, and we aim to publish Version 2.0 on
January 1, 2014.

In the mean time, we are also building on what we learned to
transition from an innovative theory-driven research project to a
sustainable and replicable peer produced problem solving accelerator.

\subsection{What we learned in 2012}

We've identified several basic and more elaborated patterns (in the
sense of \cite{Origins}, \cite{Tales}, \cite{vlissides1995design})
that describe ``the peeragogy effect''.

The central pattern is that of a Roadmap, which can apply at the
individual level, as a personal learning plan, or at a project
level.\footnote{See ``What are Learning Analytics?'' by George Siemens
  \url{http://www.elearnspace.org/blog/2010/08/25/what-are-learning-analytics/}
  for more on individual learner profiles as a context for developing
  and tracing progress on individual learning plans.}  The roadmap may
include a reason ``why'' \cite{sinek2009start}, exposition about the
goal, indicators of progress, a section for future work, and so forth.
Our initial roadmap for the project was the preliminaly outline of the
handbook; as the handbook approached publishability, we spun off
additional goals into a new roadmap for developing further aspects of
the project.\footnote{\url{http://peeragogy.org/peeragogy-org-roadmap/}}

Group culture around maintaining and updating the roadmap define
project governance.  Other patterns flesh out the project's emergent
properties in a sort of ``agora'' of possibilities.

\subsection{Patterns} \label{patterns}

\vspace{.2in}
\hspace{.2in}
\begin{minipage}{.4\textwidth}
\begin{description}
\item[Roadmap] Plans for future work, direction towards a goal, dynamic
\item[Convene] \quad \\[-.1in]
\begin{itemize}
\item[\emph{Project}] Most projects involve learning!
\item[\emph{Guide}] Overviews expose the lay of the land collecting content
  and stories.
\end{itemize}
\end{description}
\end{minipage}

\hspace{.2in}
\begin{minipage}{.4\textwidth}
\begin{description}
\item[Organize] \quad \\[-.1in]
\begin{itemize}
\item[\emph{Roles}] Specialize and mix it up. Play to participants' strengths and skills.
\item[\emph{Newcomer}] Create a guide for ``beginner's mind'' and help avoid
  need to bring new members up to speed each ``meeting.''
\item[\emph{Wrapper}] Consolidate and contain. Front end appearance to
  participants.
\end{itemize}
\end{description}
\end{minipage}

\hspace{.2in}
\begin{minipage}{.4\textwidth}
\begin{description}
\item[Cooperate] \quad \\[-.1in]
\begin{itemize}
\item[\emph{Heartbeat}] The ``heartbeat'' of the group keeps energy flowing.
\item[\emph{Capacity}] Know your limits, find ways to get other people
  involved.
\item[\emph{Moderation}] When leaders step back, dynamics can improve;
  moderator serves as champion and editor.
\item[\emph{Poll}] Invite brainstorming, collecting ideas, questions, and
  solutions.
\item[\emph{Patience}] Do not hold grudges and do not make pithy comments
  about team members or other institutions.
\end{itemize}
\end{description}
\end{minipage}

\hspace{.2in}
\begin{minipage}{.4\textwidth}
\begin{description}
\item[Assess]  \quad \\[-.1in]
\begin{itemize}
\item[\emph{Reuse}] Repurposing, tinkering, or creating from scratch?
\item[\emph{Discern}] Found a pattern? Give it a title and example.
\item[\emph{Believe}] Participants need to buy in to the idea or philosophy behind a project.
\item[\emph{Sacrifice}] Participants must be willing to sacrficice credit.
\end{itemize}
\end{description}
\end{minipage}

These patterns provide a natural framework for participatory design
and research in peer learning projects.  For example, can use them to
provide a terse profile for our project (Figure \ref{catalog}).

\begin{figure}[h]
\begin{cframed}[black]
{\bf Roadmap}: We aim to build a generally Wikipedia-like project for
learning.  We different from Wikiversity and P2PU because we're not
focusing on courses or learning materials, but on real-time support
for individual projects.  In this sense we are more like the
historical university. \\

\emph{Guide}: The Handbook is our guide for ourselves and newcomers. \\

\emph{Wrapper}: Christopher Tillman Neal has provided Weekly or
Bi-Weekly summaries throughout the duration of the project. \\

\emph{Heartbeat}: Charlotte Piece has hosted regular collaborative
editing sessions using Google Hangouts and Google Docs. \\

\emph{Capacity}: The Social Media Forum is accessible only to logged in
users.  We've also made routine efforts to ``prune'' the list of
contributors by asking people to actively renew interest or leave.
This gives us a core group of active contributors. \\

\emph{Poll}: Google+ has been a place to bring in new ideas and weak
links to other projects. \\

\emph{Reuse}: We have not had to build any special-purpose software to
run the project.  \\

\emph{Believe}: Often, people who \emph{like} the project are not the
strongest contributors, but it is rewarding to have enthusiastic
readers.  We've worked hard to make it easy for people to ``convert''
into contributors; there is qualitatively a missionizing aspect to the
project, and we have drawn on previous work in this space to refine
our approach, particularly with regard to the efficiency of bringing
in groups rather than individuals \cite{Bridges}.
\end{cframed}
\vspace{-.2in}
\caption{Terse description of the Peeragogy project using our pattern catalog \label{catalog}}
\end{figure}

\subsection{Survey} \label{survey}

In an effort to document "the paths in the grass" \cite{Wall} that come from unexpected links between different things in our successful publication of last year's Handbook we prepared a short survey for Peeragogy project participants. We asked people how they had participated (e.g Signing up for access to the Social Media Classroom and mailing list, Joining the Google+ Community, Authoring articles, etc.), and what goals or interests motivated their participation. We asked them to describe the Peeragogy project itself in terms of its aims and to evaluate its progress over the first year of its existence. As another measure of ``investment'' in the project, we asked, with no strings attached, whether the respondent would consider donating to the Peeragogy project. This survey was circulated to 223 members of the Peeragogy Google+ community, both en masse, and with individual +{\sl Name} call-outs, as well as to the currently active members of the Peeragogy mailing list.

The outline of the project's purpose ranged from the general: ``How to
make sense of learning in our complex times'' - to much more
specific:

\begin{quote}
``Push education further, providing a toolbox and [techniques] to
self-learners. In the peeragogy.org introduction page we assume that
self-learners are self-motivated, that may be right but the Handbook
can also help them to stay motivated, to motivated others and to face obstacles that may erode motivation.''
\end{quote}

Considering motivation as a key factor, it is interesting to observe
how various understandings of the project's aims and its flaws
intersected with personal motivations. For example, one respondent
(who had only participated by joining the Google+ community) was:
``[Seeking] [i]nformation on how to create and engage communities of
interest with a shared aim of learning."

More active participants justified their participation in terms of what they get out of taking an active role, for instance:

\begin{quote}
``Contributing to the project allows me to co-learn, share and co-write ideas with a colourful mix of great minds. Those ideas can be related to many fields, from communication, to technology, to psychology, to sociology, and more.''
\end{quote}

The most active participants justified their participation in terms of
beliefs or a sense of mission:

\begin{quote}
``Currently we are witnessing many efforts to incorporate technology as
an important tool for the learning process. However, most of the
initiatives are reduced to the technical aspect (apps, tools, social
networks) without any theoretical or epistemological
framework. Peeragogy is rooted in many theories of cooperation and
leads to a deeper level of understanding about the role of technology
in the learning process. I am convinced of the social nature of
learning, so I participate in the project to learn and find new
strategies to learn better with my students.''
\end{quote}

Or again:

\begin{quote}
``I wanted to understand how "peer production" really works. Could we
create a well-articulated system that helps people interested in peer
production get their own goals accomplished, and that itself grows and
learns? Peer production seems linked to learning and sharing - so I
wanted to understand how that works.''
\end{quote}

They also expressed criticism of the project, implying that they may feel rather powerless to make the changes that would correct course:

\begin{quote}
``Sometimes I wonder whether the project is not too much `by education specialists for education specialists'. I have the feeling peer learning is happening anyway, and that teens are often amazingly good at it. Do they need `learning experts' or `books by learning experts' at all? Maybe they are the experts. Or at least, quite a few of them are.''
\end{quote}

Another respondent was more blunt:

\begin{quote}
``What problems do you feel we are aiming to solve in the Peeragogy
project? We seem to not be sure. How much progress did we make in the
first year? Some..got stuck in theory.''
\end{quote}

But, again, it's not entirely clear how the project provides clear pathways for contributors to turn their frustrations into changed behavior or results. Additionally we need to be entirely clear that we are indeed paving new ground with our work. If there are proven peer learning methods out there we have not examined and included in our efforts, we need to find and address them. Peeragogy is not about reinventing the wheel.

It's also not entirely clear whether excited newbies will find pathways to turn their excitement into shared products or process. For example, one respondent (who had only joined the Google+ community) had not yet introduced their fascinating projects publicly:

\begin{quote}
``I joined the Google+ community because I am interested in developing peer to peer environments for my students to learn in. We are moving towards a community-based, place-based program where we partner with community orgs like our history museum for microhistory work, our local watershed community and farmer's markets for local environmental and food issues, etc. I would love for those local efforts working with adult mentors to combine with a peer network of other HS students in some kind of cMOOC or social media network.''
\end{quote}

Responses such as this highlights our need to make ourselves available to hear about exciting new projects from interested peers, simultaneously giving them easier avenues to share. Our work on developing a peeragogy accelerator in the next section is an attemt to address this situation.

\subsection{Survey Analysis} \label{survey-analysis}

Clearly, many participants will have intermediate levels of investment -- basically ``social consumers'' of the project as a ``product''. However, if we think about the metaphor of a college or university, this description also applies to most members of the student body, who are physically present only for a brief portion of the institution's history, and who may not join the student government etc. A blueprint for a distributed university or peers we will discuss briefly in the conclusion perhaps named ``Peeragogy.EDU''\footnote{\url{http://peeragogy.org/knight-foundation-prototype-fund-proposal-unfunded/}} should include plenty of room for people who take a less involved role. Nevertheless, integrating some of these light-weight contributions (like blog posts about the ideas) would be an important role for more active participants to take on.

Some of the big challenges can be parsed out using our high-level patterns. First, in our work, we uncovered pretty much everyone needs a team. Building off that, these issues could define new project ``roadmaps'' and ``teams'':

\begin{description}
\item[Cooperate] ~ How to build a really strong collaboration?

``A team is not a group of people who work together. A team is a group of people who trust each other...''

\item[Convene] ~ How to build a more practical focus?

``The insight that the project will thrive if people are working hard on their individual problems and sharing feedback on the process seems like the key thing going forward. This feels valuable and important.''

\item[Organize] ~ How to connect with motivated newcomers?

``I just came on board a month ago. [...] I am designing a self organizing learning environment (SOLE) or PLE/PLN that I hope will help enable communities of life long learners to practice digital literacies.''

\item[Assess] ~ How to be effective and relevant?

``I am game to also explore ways to attach it to spaces where funding can flow based on real need in communities.''
\end{description}

The basic workflow in the project -- discussing ideas, condensing them into written handbook sections, annotating them with new resources in informal discussion, reevaluating and rewriting -- seems relatively sustainable, as long as we have community members and editors who are willing to do the work. However, the long-term relevance of the project will come from building workflows that are less self-referential and more applied. With the current foundation in theory and examples from literature and day-to-day activities of participants, we are prepared to be a more effective peer-produced accelerator for peer learning and peer production projects. The kinds of critical questions elaborated above would tend to apply to other projects as well.

It is worth comparing the results from this survey with the results of an earlier pilot, in which we surveyed the then-current body of participants with questions inspired by Boud and Lee's paper on ``peer learning as a pedagogic discourse for research education'' \cite{boud2005peer} and by the After Action Review (see Section \ref{PAR} of this paper).\footnote{\url{http://peeragogy.org/organize/}}
\begin{itemize}
\item {\bf Slow formation of ``peer'' relationships.}  Many early respondents did not feel they were getting to know one another as peers.  This has changed for the core group (including current co-authors), but there is a broad periphery with much weaker ties.
\item {\bf ``Co-learning'', ``co-teaching'', ``co-producing''?}  One respondent in the early survey wrote, ``the question is, are we learning from others by ourselves or are we [...] helping others to learn?''   Many participants seem to have arrived in the project expecting to get something from outside -- fewer expect contribute. 
\item {\bf Weak structure at the outset, versus a more flexible approach}  In the pilot survey, some respondents indicated that confusion should be expected in peeragogy.  One proposed ``solution'' would have been to ``have had a small group of people as a cadre that had met and brainstormed before the first live session [...] tasked [us with] roles [and gotten us onto] the same page.''  However, such an approach would have brought us closer to the ``provisionist'' approach critiqued by Boud and Lee.  The messy and emergent unplanned structure in the Peeragogy project may have been one of its greatest assets, even if it was frustrating to work with.
\item {\bf Technological concerns}  Early on in the project, we wondered ``How might different platforms handle the tension between `conversations' and `content production'?''  This question has since been resolved in favor of multiple technologies used for different purposes.  A more recent concern has been that technical experiments can alienate participants who may be less clear about where to engage. 
\item {\bf Sample size and response rate}  The early survey had around 9 respondents, although not all respondents answered all of the questions.  Although the sample size has grown, the \emph{response rate} remains approximately the same.  This suggests that we have successfully ``scaled'' the project, although this itself can be seen as a mixed success (see Section \ref{conclusion}).
\end{itemize}

\section{The Peeragogy Accelerator} \label{accelerator}

Having piloted the peeragogy patterns as a research method for
understanding the peeragogy project itself, work is now underway to
apply them in other peer learning and peer production communities.

We wish to emphasize that the transition from ``innovative project in sustainable learning design'' to a practical peer-produced accelerator is natural, but not trival.   We have been doing ``peer
support'' and ``critical thinking'' all along, but we still have a lot to learn about how to do this effectively.

\begin{paragraph}{Example}
Joe and Charlie made a simple peer support pact outside of P2PU to sit
in on each others courses $\rightarrow$ this led to discussions $\rightarrow$ which led to papers $\rightarrow$ which produced some of the seeds for the Peeragogy project.
\end{paragraph}

\begin{paragraph}{What makes Peeragogy different?}
\begin{itemize}
\item we're not just offering content
\item we're also not offering "classes" (or a place to organize classes)
\item we're working at a higher level, more strategic
\item we're focusing on people instead of topics/subjects;
\item if we have a common topic, it is something like ``leading by example in distributed teams.''
\end{itemize}
\end{paragraph}

\subsection{Pilot: ``PeerPub-U''}

Drawing on the experience and skills of its 108+ members, Independent
Publishers of New England's (IPNE) plans to build an open learning and support network (a.k.a. PeerPubU, or "Peer Publishing University") to address issues in independent publishing and provide education to its members - part of its stated mission. IPNE facilitators hope that this network will dovetail with other planned membership-building efforts and help raise the standards and maximize success of indie publishers in New England in this fast-changing business sector.

On Jan. 26, 2013, regional independent publishers and authors attended
the IPNE.org greater Boston branch meeting in Arlington, MA, billed as
a ``plenary session'' for PeerPubU. In addition to the live in-person
meeting, Peeragogy Handbook team members Gigi Johnson (Los Angeles,
US), Roland Legrand (Antwerp, Belgium) and Anna Keune (Helsinki,
Finland) joined via Google+ Hangout.  Plenary meeting observations:

At the meeting, IPNE President Tordis Isselhardt suggested a Peeragogy-style effort might more likely meet with success if the organization's members rallied around a specific project, perhaps an "Independent Publishing Handbook," (like the Peeragogy Handbook) in addition to creating resource repositories and sharing of expertise as individuals' challenges arise.

While brainstorming ways to sustain motivation, it was suggested that the 108 members of the association could earn authorship credit for contributing articles; editor credit for working on the manuscript; and could spin off their own chapters as stand-alone, profit-making publications. Members agreed to set up the project in IPNE's BaseCamp content-management platform. Members who express interest at the branch meetings are regularly added. There were 12 self-selected participants in BaseCamp by March, 2013.

Members attending the plenary session were a little disconcerted to learn that there would be co-facilitators, but not an overall leader, of the PeerPub-U. Potential particpants were encouraged to prepare by visiting peeragogy.org and
extracting practices and patterns that might work best for IPNE and to ask the "Big Five" questions of themselves.

IPNE officers at the meeting perked up at the suggestion that "PeerPubU" might become a case study in future editions of The Peeragogy Handbook.

Live, in-person development sessions of the pilot Peeragogy project take place at IPNE Greater Boston Regional Branch meetings on a monthly basis, and project facilitator Charlotte Pierce is researching online collaborative publishing platforms like WriteLaTeX and Scrivener. Google+ Hangouts or Skype will be used for live meetings.

\subsection{PlanetMath}

Team member Joe Corneli is applying the Peeragogy pattern catalog to analyse the socio-technical changes that have emerged with the launch of new beta software for a decade-old mathematics website \cite{corneli-thesis}.  The aim is to transform a peer produced mathematics reference work into a peer produced mathematics learning environment.  In the mid- to long-term, this will connect with questions about how to generate revenue for the site.

\subsection{The Uncertainty Principle}

Team member Charlie Danoff looks to increase both distribution and number of contributors to his four-year-old Chicago based "zine" (an independently published magazine) The Uncertainty Principle\footnote{\url{http://theup.biz}}. He is looking for peers in the zine community in Chicago\footnote{\url{http://chicagozinefest.org/}} and across the world for case studies to learn this medium's best practices. He will then combine what he learns with Peeragogy patterns with his own team and new members.

\subsection{Bergamo}

Team member Fabrizio Terzi is seeking to put Peeragogy in Action via a
HUB-inspired\footnote{\url{http://www.the-hub.net/}} Laboratory
Project Art \& Open Technology Incubator.  He is drawing on
peeragogical patterns and his peer learning network to develop an
application to submit to the Bergamo, Italy Art \& Tourism Council.

\section{Conclusion} \label{conclusion}

(I suggest to two sub-sections for the conclusion - first the theoricial side with present and future and second the practical side - Regis)

This paper can be thought of as a moment in time of the Peeragogy Project, sort of like a financial statement, or balance sheet. We can use it to understand where the our time and treasure been going over the past year in order to rationally allocate our efforts for the next phase.

On a theorical side, ...(short resume of the findings)

Going forward we need to record data better -> (discussion with
R\'egis about
this)\footnote{\url{https://plus.google.com/u/0/101437188321463196206/posts/eUxkDqEAmvG}}
-> can we move towards some sort of learning achieved by hour metric?


I had an interesting idea about the power law of participation. The 90/10/1 breaks down when you work with small groups (like 1 person) -- and I'm guessing that the ``long tail'' is much longer (and skinnier) in G+ than in SMC. However, it's not necessarily that fat in SMC either.

To better reach our goal and reinforce our findings, future work should investigate the raw data gathered while writing the first version of the peeragogy handbook. By that we mean exploring the Social Media Classroom, Wordpress and Google+ statistics of use and content to confront them with the following hypotheses:

\paragraph{Hypothesis 1}
In a peeragogy environment the 90-9-1 principle which states that more people lurk rather than participate doesn't apply.

\paragraph{Hypothesis 2} The long tail theory applies in the sense that a significant amount of participants may contribute only with a few amount of ideas, even just one. (is it correct to use this theory for ideas as it is more often used to talk about products in marketing? - Regis)

\paragraph{Hypothesis 3} Most contributors influence the structure of the site not only the content (meaning in the case of a wiki re-organize the links over time). In the case of the wiki, it would be interesting to reprensent the evolution of the site in a mindmapping way showing the pages and their links to other pages.

\paragraph{Hypothesis 4} Peer learning projects need to find the right balance of freedom, interest and bureaucracy.


On a practical side the handbook can be valuable for some professionals. For example, programmers may find the techniques we have developed useful for "peer-sourcing" their documentation. Other examples?

It's also an effective way for the members of the peeragogy project to come together at this point in the project to work on the next phase.

The phenomenological approach we've taken in this paper is justified given the fact that throughout our efforts, ``quality'' has been more important as ``quantity''.  As a way to sum up the paper and look ahead to future work, we present the following jointly-written Review.

\subsection{Paragogical Action Review} \label{PAR}

This aspect of the paper would harkens back to the paper on paragogy that Joe and Charlie presented at OKCon 2011 \cite{paragogy}, further refined with our experience at e/merge 2012.  Thinking longitudinally, we can ask: have we learned anything since ``our P2PU days?''  The Paragogical Action review is a revision to the US Army's ``After Action Review'' \cite{armyXtrainingX2002}; in our case the focus is on understanding peer learning work-in-progress.

\paragraph{Review what was supposed to happen}
Thought to act, think and act for inspire the learning experience.  Even a student without any particular background or special skills is able to find the right people on the network.  With the Internet and other advances, the technology exists for a new paradigm of creation, one where anyone can be an artist or became an "expert" and anyone can succeed, based not on their industry or academyc connections, but on their merit. Peeragogy creates the right conditions for the production of Open Educational Resources (OER), freely accessible, usually openly licensed documents and media that are useful for teaching, learning, educational, assessment and research purposes. Although some people would consider the use of an open format to be an essential characteristic of OER this is not a universally acknowledged requirement.
\paragraph{What happened / is happening?}
We made a great open educational resource we are creating a network of people.  Our Open Educational Resources include: full courses, course materials, modules, learning objects, open textbooks, streaming videos, tests, software, and any other tools, materials, or techniques used to support access to knowledge. Resources for the implementation of open education include intellectual licenses that govern open publishing of materials, design-principles, and localization of content.  OER are intended to be available for a variety of educational purposes. Unlike degree granting accredited institutions, OER neither award degrees nor provide academic or administrative support to students seeking college credits towards a degree


\paragraph{What is right or wrong with what we're doing / have done?}
Quite cool project, shows how to put things into action, but it's more important to have a model that supports your ideas.  In practice we don't always know where we're going!  Everything is constantly in change, and this is quite cool. 

\paragraph{What did we learn or change?}

We're learning every day.  If we can state one critical take-away from the peeragogy handbook, it is that ``knowledge is only valid when it can be applied.''  

\paragraph{How can we learn from this experience to improve next time?}

It is a dynamic learning, where the only constant is the change of personal respective Educational associations and interests.  We can make our own accellerator that will work anywhere, because we're on the network.  Probably it will be important to find a common goal.  Practice is great, but we need to build our skills (e.g. with grant applications) to be able to successfully capture opportunities as they come up.

%
% The following two commands are all you need in the
% initial runs of your .tex file to
% produce the bibliography for the citations in your paper.
% (Uncomment to reproduce the .bbl file if needed!)

% \bibliographystyle{abbrv}
% \bibliography{bib}  % sigproc.bib is the name of the Bibliography in this case

\begin{thebibliography}{10}

\bibitem{Origins}
C.~Alexander.
\newblock The origins of pattern theory, the future of the theory, and the
  generation of a living world.
\newblock In {\em {A}{C}{M} {C}onference on {O}bject-{O}riented {P}rograms,
  {S}ystems, {L}anguages and {A}pplications ({O}{O}{P}{S}{L}{A})}, San Jose,
  California, 1996.
\newblock Keynote address.

\bibitem{armyXtrainingX2002}
U.~Army.
\newblock Training the force {(Field} manual no. 7-0), Oct 2002.

\bibitem{bacon2012art}
J.~Bacon.
\newblock {\em The art of community: Building the new age of participation}.
\newblock O'Reilly Media, 2012.

\bibitem{boud2005peer}
D.~Boud and A.~Lee.
\newblock `{P}eer learning' as pedagogic discourse for research education.
\newblock {\em Studies in Higher Education}, 30(5):501--516, 2005.

\bibitem{corneli-thesis}
J.~Corneli.
\newblock {\em Peer supported problem solving and mathematical knowledge}.
\newblock PhD thesis, The Open University, 2013.

\bibitem{paragogy}
J.~Corneli and C.~J. Danoff.
\newblock Paragogy.
\newblock In S.~Hellmann, P.~Frischmuth, S.~Auer, and D.~Dietrich, editors,
  {\em Proceedings of the 6th Open Knowledge Conference}, Berlin, Germany,
  2011.

\bibitem{PeeragogyinAction}
J.~Corneli, A.~Keune, C.~J. Danoff, and A.~Lyons.
\newblock {P}eeragogy in {A}ction.
\newblock In {\em The Open Book}, pages 80--87. The Finnish Institute in
  London, 2013.

\bibitem{crowstonXdefiningX2003}
K.~Crowston, H.~Annabi, and J.~Howison.
\newblock Defining open source software project success.
\newblock In {\em Proceedings of the 24th {I}nternational {C}onference on
  {I}nformation {S}ystems ({I}{C}{I}{S} 2003)}, pages 327--340. Citeseer, 2003.

\bibitem{Wall}
A.~Dougherty.
\newblock Gluing the web together: An interview with {L}arry {W}all.
\newblock {\em ZD Internet User}, 1998.

\bibitem{frayling1993research}
C.~Fraylings.
\newblock {\em Research in art and design}.
\newblock Royal College of Art London, 1993.

\bibitem{Tales}
R.~P. Gabriel.
\newblock {\em Patterns of Software}.
\newblock Oxford University Press, New York, 1996.

\bibitem{Bridges}
D.~McGavran.
\newblock {\em The Bridges of God}.
\newblock World Dominion Press, 1955.

\bibitem{Page2008difference}
S.~Page.
\newblock {\em The Difference: How the Power of Diversity Creates Better
  Groups, Firms, Schools, and Societies (New Edition)}.
\newblock Princeton University Press, 2008.

\bibitem{OpenAdvice}
L.~Pintscher, editor.
\newblock {\em Open Advice}.
\newblock lulu.com, 2012.

\bibitem{sinek2009start}
S.~Sinek.
\newblock {\em Start with why: How great leaders inspire everyone to take
  action}.
\newblock Portfolio, 2009.

\bibitem{tomlinson2012massively}
B.~Tomlinson, J.~Ross, P.~Andre, E.~Baumer, D.~Patterson, J.~Corneli,
  M.~Mahaux, S.~Nobarany, M.~Lazzari, B.~Penzenstadler, et~al.
\newblock Massively distributed authorship of academic papers.
\newblock In {\em Proceedings of the 2012 ACM annual conference extended
  abstracts on Human Factors in Computing Systems Extended Abstracts}, pages
  11--20. ACM, 2012.

\bibitem{vlissides1995design}
J.~Vlissides, R.~Helm, R.~Johnson, and E.~Gamma.
\newblock Design patterns: Elements of reusable object-oriented software.
\newblock {\em Reading: Addison-Wesley}, 49, 1995.

\end{thebibliography}
%%%%%%%%%%%%%%%%%%%%%%%%%%%%%%%%%%%%%%%%%%%%%%%%%%%%%%%%%%%%%%%%%%%%%%%%%%%%%%%%%%%%%% Cut here

\end{document}


